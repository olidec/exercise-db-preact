
\section{Einleitung}

Diese Arbeit hatte zum Ziel eine Webapplikation zu kreieren, auf der man Aufgaben kreieren, aussuchen und anschliessend herunterladen kann. Wie man sich vorstellen kann ist dies ein grosses Projekt mit Potenzial zum endlosen Ausbau. Aus diesem Grund haben wir uns in unserer Vereinbarung auf folgende Grundfunktionalitäten beschränkt\footnote{"beschränkt" war immer noch komplex genug.}:
\begin{itemize}
  \item Suchen und Filtern von Aufgaben aus der Datenbank, die in einer Vorschau (z.B. via KaTeX) in der Applikation dargestellt werden
  \item Speichern von gewünschten Aufgaben in einer Art Warenkorb, von wo aus sie anschliessend im PDF oder im LaTeX Format heruntergeladen werden können
  \item Login Maske, damit die Aufgaben nur registrierten Benutzern (Lehrpersonen) zugänglich sind
  \item Benutzerkonto Administration (Passwort ändern, Konto löschen etc.)
  \item Formular zum Erstellen von eigenen Aufgaben in die Datenbank.
\end{itemize}

Im Grossen und Ganzen ist es uns gelungen diese Funktionalitäten zur Verfügung zu stellen. Einzig die Administration des Benutzerkontos steht -- in der vorliegenden Fassung -- nicht zu Verfügung. Aufgrund des drohenden "feature creeps" wurde beschlossen, auf diese Funktionalität momentan zu verzichten. Personenadministration kann -- auf Anfrage -- von einem berechtigten Benutzer über den server direkt erledigt werden.

Das Arbeiten an diesem Projekt erforderte viel Koordination und das Erlernen von neuen Ideen und Konzepten.  Durch dieses Projekt sind wir einer Vielzahl von Techniken und Begriffen begegnet: Server, Client, Datenbank, Full-Stack, Express, JavaScript, React, Preact, Docker, \LaTeX, Authentifizierung, Zertifikate etc. Einige kennen wir nun sehr gut, andere einfach gut genug, um damit zu arbeiten. Es war auf jeden Fall nicht möglich in all diesen Bereichen bis zum Expertentum vorzudringen. Weil wir aber auf so vielen verschiedenen Ebenen gearbeitet haben, haben wir einen einzigartigen Einblick in die komplette Bau- und Funktionsweise einer Webapplikation erhalten. Diese Erfahrungen und das erlernte Wissen, wird uns sicher in unserer Arbeit als Informatik-Lehrpersonen helfen.

Das Endprodukt dieser Arbeit ist eine full-stack Webapplikation, die auf einem Linode Server läuft. Die Applikation läuft in einem Docker Container und ist durch ein selbst-gezeichnetes Zertifikat gesichert.

Die Applikation kann unter der URL \url{https://letstalkaboutx.ch} aufgerufen werden und kann und soll frei genutzt werden.