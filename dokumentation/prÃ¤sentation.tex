\documentclass{beamer}
\usepackage[utf8]{inputenc} 
\usepackage[ngerman]{babel} % Passt alle automatischen Texte an. 
%\usepackage[english]{babel} % Diese Variante verwenden, fall man eine Arbeit auf Englisch schreibt.

\parindent=0cm
\parskip=0.3cm
\linespread{1.5}

\usepackage{graphicx}
\graphicspath{{images/},{../images/}}

\newcommand{\centfig}[2]{\begin{center}
  \includegraphics[width=#1\textwidth]{#2}
  \end{center}}




% Ein paar standard Pakete. Man muss genau wissen, wofür alle gebraucht werden...
\usepackage{amsmath}
\usepackage{amsfonts}
\usepackage{amssymb}
\usepackage{pdfpages}
\usepackage{datetime}
\usepackage{float}
\usepackage[utf8]{inputenc}
%Code
\usepackage{listings}
\usepackage{xcolor}


\lstset{
    language=Python,
    basicstyle=\ttfamily\scriptsize,
    keywordstyle=\color{blue},          % Schlüsselwörter (z.B. function, if, else)
    stringstyle=\color{orange},         % Zeichenketten
    commentstyle=\color{gray},          % Kommentare
    numberstyle=\tiny\color{gray},      % Zeilennummern
    stepnumber=1,                       % Zeilennummernschritt
    numbersep=10pt,                     % Abstand zu Zeilennummern
    showspaces=false,                   % Leerzeichen nicht anzeigen
    showstringspaces=false,             % Leerzeichen in Zeichenketten nicht anzeigen
    breaklines=true,                    % Zeilenumbruch bei langen Zeilen
    frame=single,                       % Rahmen um den Code
    tabsize=2,                          % Tab-Breite
    captionpos=b,                       % Position der Beschriftung (b=unten)
    morekeywords={console, log},        % Zusätzliche Schlüsselwörter
    keywordstyle=[2]\color{purple},     % Stil für spezielle Schlüsselwörter (z.B. console, log)
    identifierstyle=\color{teal},       % Stil für Variablen und Funktionsnamen
    numberstyle=\tiny\color{gray},      % Stil für Zeilennummern
    emph={function, return, var},       % Hervorhebung spezifischer Schlüsselwörter
    emphstyle=\color{magenta},          % Stil der hervorgehobenen Wörter
    commentstyle=\color{gray}, % Stil für Kommentare (z.B. grünlich)
    literate=
    *{0}{{{\color{red}0}}}{1}           % Zahlen in Rot
     {1}{{{\color{red}1}}}{1}
     {2}{{{\color{red}2}}}{1}
     {3}{{{\color{red}3}}}{1}
     {4}{{{\color{red}4}}}{1}
     {5}{{{\color{red}5}}}{1}
     {6}{{{\color{red}6}}}{1}
     {7}{{{\color{red}7}}}{1}
     {8}{{{\color{red}8}}}{1}
     {9}{{{\color{red}9}}}{1}
     {=>}{{{\color{blue}=>}}}{2},    
     comment=[l]{//}  % Legt // als Kommentarzeichen fest   % Operatoren hervorheben
}


\title{Webapplikation zur Erstellung und Ausgabe von Mathematikaufgaben}
\subtitle{Abschlussprojekt GymInf Weiterbildung}
\author{Oliver De Capitani und Patrick Weber}
\date{\today}

\begin{document}

% Titelblatt
\frame{\titlepage}

% Inhaltsverzeichnis
% \section*{Inhaltsverzeichnis}
% \begin{frame}{Inhaltsverzeichnis}
%     \tableofcontents
% \end{frame}

% Kapitel: Einleitung
\section{Einleitung}
\begin{frame}{Einleitung}
    \begin{itemize}
        \item<1-> Ziel: Entwicklung einer Webapplikation, um Mathematikaufgaben zu erstellen, zu verwalten und herunterzuladen.
        \item<2-> Inspiration: Frühere Plattform \url{munterbunt.ch}.
        \item<3-> Anforderungen:
        \begin{itemize}
            \item Benutzerregistrierung und Login.
            \item Aufgaben filtern, sammeln und im Tex-Format herunterladen.
            \item Eigene Aufgaben erstellen, bearbeiten oder löschen.
        \end{itemize}
        \item<4-> Technologie-Stack: Full-Stack-Entwicklung mit Docker, React/Preact, Prisma und PostgreSQL.
    \end{itemize}
\end{frame}

% Kapitel: Struktur der Webapplikation
\section{Struktur der Webapplikation}
\subsection{Architekturübersicht}
\begin{frame}{Architekturübersicht}
    \centfig{0.7}{docker-container.png}
    % \begin{itemize}
    %     \item Anwendung läuft in einem Docker-Container.
    %     \item Proxy vermittelt zwischen Client, Server und Datenbank.
    %     \item Sicherheit: HTTPS über selbstsignierte Zertifikate.
    % \end{itemize}
\end{frame}

\begin{frame}{Architekturübersicht}
    % \centfig{0.5}{docker-container.png}
    \begin{itemize}
        \item<1-> Anwendung läuft in einem Docker-Container.
        \item<2-> Proxy vermittelt zwischen Client, Server und Datenbank.
        \item<3-> Sicherheit: HTTPS über selbstsignierte Zertifikate.
    \end{itemize}
\end{frame}

\subsection{Technische Details}
\begin{frame}{Technische Details}
    \begin{itemize}
        \item<1-> \textbf{Server:}
        \begin{itemize}
            \item Express.js zur Verarbeitung von Anfragen.
            \item Passport.js für Authentifizierung.
            \item Prisma ORM für Datenbankoperationen.
        \end{itemize}
        \item<2-> \textbf{Datenbank:} PostgreSQL mit Prisma ORM.
        \item<3-> \textbf{Frontend:} React/Preact mit Komponentenhierarchie.
        \item<4-> \textbf{Deployment:} Docker-Container auf einem Linode-Server.
    \end{itemize}
\end{frame}

\subsection{Workflow der Anwendung}
\begin{frame}{Workflow der Anwendung}
    \centfig{0.7}{web-app-workflow.png}
    % \begin{enumerate}
    %     \item Nutzeranfrage (z. B. Download einer Aufgabe) wird an den Server gesendet.
    %     \item Server fragt die Datenbank ab und verarbeitet die Ergebnisse.
    %     \item Die resultierenden Daten werden an den Client zurückgegeben.
    % \end{enumerate}
\end{frame}

\begin{frame}{Workflow der Anwendung}
    % \centfig{0.6}{web-app-workflow.png}
    \begin{enumerate}
        \item<1-> Nutzeranfrage (z. B. Download einer Aufgabe) wird an den Server gesendet.
        \item<2-> Server fragt die Datenbank ab und verarbeitet die Ergebnisse.
        \item<3-> Die resultierenden Daten werden an den Client zurückgegeben.
    \end{enumerate}
\end{frame}

% Kapitel: Datenbank
\section{Datenbank}
\begin{frame}{Datenbankstruktur}
    \begin{itemize}
        \item<1-> Tabellen:
        \begin{itemize}
            \item \textbf{User:} Speichert Benutzerdaten und Login-Informationen.
            \item \textbf{Exercise:} Enthält Aufgabeninhalte und Metadaten.
            \item \textbf{Category/Subcategory:} Hierarchische Organisation von Aufgaben.
        \end{itemize}
        \item<2-> Beziehungen:
        \begin{itemize}
            \item 1:n Beziehung zwischen Benutzer und Aufgaben.
            \item n:1 Beziehung zwischen Aufgaben und Kategorien.
        \end{itemize}
    \end{itemize}
\end{frame}

% Kapitel: Client-Architektur
\section{Frontend-Architektur}
\begin{frame}{Frontend-Architektur}
    \begin{itemize}
        \item<1-> Komponenten:
        \begin{itemize}
            \item \textbf{AufgabenSuchen:} Filterung und Anzeige von Aufgaben.
            \item \textbf{AufgabeDetails:} Detaillierte Ansicht einer Aufgabe.
            \item \textbf{Warenkorb:} Verwaltung gesammelter Aufgaben.
            \item \textbf{AufgabeHinzufügen/Ändern:} Formulare zur Bearbeitung.
        \end{itemize}
        \item<2-> Zustandshandling:
        \begin{itemize}
            \item \texttt{useContext}: Globaler Zustand für Suche und Warenkorb.
            \item \texttt{useState}, \texttt{useEffect}: Lokale Zustandsverwaltung.
        \end{itemize}
    \end{itemize}
\end{frame}

% Kapitel: Prozess und Herausforderungen
\section{Prozess und Herausforderungen}
\subsection{Koordination}
\begin{frame}{Koordination}
    \begin{itemize}
        \item Nutzung von Trello zur Planung und Aufgabenverteilung.
        \item Agile Kommunikation im kleinen Team.
        \item Herausforderung: Übersichtlichkeit und Priorisierung in der Anfangsphase.
    \end{itemize}
    \centfig{0.6}{trello.png}
\end{frame}

\subsection{Probleme und Lösungen}
\begin{frame}{Probleme und Lösungen}
    \begin{itemize}
        \item<1-> \textbf{CORS-Probleme:}
        \begin{itemize}
            \item Ursache: Anfragen an unsichere Domains.
            \item Lösung: Spezifikation erlaubter Domains im Server-Code.
        \end{itemize}
        \item<2-> \textbf{Testdatenbank:}
        \begin{itemize}
            \item Ursache: Konflikte durch manuell gesetzte IDs.
            \item Lösung: Automatische ID-Zuweisung.
        \end{itemize}
    \end{itemize}
\end{frame}

% Kapitel: Zusammenfassung und Ausblick
\section{Zusammenfassung und Ausblick}
\begin{frame}{Zusammenfassung}
    \begin{itemize}
        \item<1-> Umsetzung der Kernfunktionen:
        \begin{itemize}
            \item Aufgaben filtern, sammeln, herunterladen.
            \item Benutzerregistrierung und Login.
        \end{itemize}
        \item<2-> Lernprozess: Vertieftes Verständnis von Full-Stack-Entwicklung.
    \end{itemize}
\end{frame}

\begin{frame}{Ausblick}
    \begin{itemize}
        \item<1-> Erweiterungen:
        \begin{itemize}
            \item Benutzerprofile und Filtermöglichkeiten.
            \item Integration weiterer Fachbereiche.
            \item Kommentarfunktion für Aufgaben.
        \end{itemize}
        \item<2-> Verbesserung der Codebasis für Skalierbarkeit.
    \end{itemize}
\end{frame}

% Abschlussfolie
\section*{Danke}
\begin{frame}{Danke!}
    \centering
    Vielen Dank für Ihre Aufmerksamkeit! \\
    \vspace{1cm}
   
\end{frame}

\end{document}
