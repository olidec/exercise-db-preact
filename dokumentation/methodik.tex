
\section{Methodik}
Unser Projekt folgte einem strukturierten Softwareentwicklungsprozess, der in mehrere Phasen unterteilt war. Wir begannen mit einer detaillierten Anforderungsanalyse und der Definition von Use Cases, um die Bedürfnisse der Endbenutzer zu verstehen.

\subsection{Backend-Entwicklung}
Die Implementierung von CRUD-Operationen und Authentifizierungssystemen stellte sicher, dass die Anwendung sicher und robust ist.

CRUD- und HTTP-Operationen:

CRUD steht für Create, Read, Update, Delete und bezieht sich auf die vier grundlegenden Operationen, die in vielen Anwendungen zur Interaktion mit Datenbanken oder Datenspeichern verwendet werden. Die Implementierung von CRUD-Operationen ermöglicht es Benutzern (oder Systemen), Daten zu erstellen, abzurufen (lesen), zu aktualisieren und zu löschen.

In einer Webanwendung werden CRUD-Operationen typischerweise über das Backend realisiert, wobei das Frontend (die Benutzeroberfläche) Anfragen an das Backend sendet, um Daten zu erstellen, abzurufen, zu aktualisieren oder zu löschen. Das Backend interagiert dann mit der Datenbank, um die angeforderten Aktionen durchzuführen und das Ergebnis (z.B. die abgerufenen Daten oder eine Bestätigung der Aktion) an das Frontend zurückzusenden, wo es dem Benutzer angezeigt wird.

Für das Backend, das oft eine API bereitstellt, können diese Operationen spezifischen Endpunkten entsprechen, z.B.:

•	GET /api/users	für das Abrufen von Benutzern (Read)

•	POST /api/user	für das Erstellen eines neuen Benutzers (Create)

•	PUT /api/user	für das Aktualisieren eines spezifischen Benutzers (Update)

•	POST /api/ex	für das Erstellen einer neuen Aufgabe (Create)

•	GET /api/ex	für das Abrufen von Aufgaben (Read)

•	PUT /api/ex/:id	für das Aktualisieren einer spezifischen Aufgabe (Update)

•	DELETE /api/ex/:id 	für das Löschen einer Aufgabe (Delete)

•	GET /api/ex/search/:searchText	für das Abrufen von Aufgaben nach Suchbegriffen (Read)

•	GET /api/ex/category/:category	für das Abrufen von Aufgaben nach Kategorien (Read)

•	GET /api/ex/:id	für das Abrufen einer spezifischen Aufgabe (Read)

•	POST /api/download/ 	für das Herunterladen der spezifischen Aufgabe im Warenkorb (Download)



\subsection{Authentifizierung und Autorisierung}
Argon2 nimmt als Eingabe das Passwort des Benutzers, einen Salz (eine zufällig generierte Zeichenfolge, die jedem Passwort-Hash hinzugefügt wird, um die Einzigartigkeit zu gewährleisten), und mehrere Parameter, die die Komplexität des Hashing-Vorgangs steuern (wie Speicherbedarf, Rechenzeit und Parallelität).Basierend auf diesen Eingaben führt Argon2 eine Reihe von komplexen, rechenintensiven Operationen durch. Diese Operationen sind so gestaltet, dass sie sowohl eine hohe Menge an CPU-Ressourcen als auch Speicher benötigen. Dies macht es sehr schwierig für Angreifer, Passwörter zu erraten oder Brute-Force-Angriffe durchzuführen, selbst wenn sie über leistungsfähige Hardware verfügen.

Das Ergebnis des Prozesses ist ein Passwort-Hash, der in der Datenbank gespeichert wird. Da der gleiche Prozess (mit dem gleichen Salz und denselben Parametern) immer denselben Hash erzeugt, kann das System das vom Benutzer bei der Anmeldung eingegebene Passwort überprüfen, ohne das eigentliche Passwort kennen oder speichern zu müssen. Sicherheit gegen Brute-Force-Angriffe: Die rechen- und speicherintensive Natur von Argon2 macht es teuer und zeitaufwändig, Hashes zu knacken. Dies schützt gegen Angriffe, die darauf abzielen, Passwörter durch Ausprobieren vieler möglicher Kombinationen (Brute-Force) zu erraten.

\subsection{Testing und Qualitätssicherung: }
Wir führten umfangreiche Tests durch, einschließlich Unit-Tests und Benutzerakzeptanztests, um sicherzustellen, dass die Anwendung den Erwartungen entspricht.


\subsection{Deployment und Inbetriebnahme: }
Nach Abschluss der Entwicklungsarbeiten wurde die Anwendung auf einer geeigneten Hosting-Plattform bereitgestellt und auf ihre Funktionalität in der Produktionsumgebung überprüft.


\section{Ergebnisse}
\subsection{Allgemeines}

Die entwickelte Webanwendung bietet Mathematiklehrern eine Plattform, auf der sie Aufgaben erstellen, durchsuchen, kommentieren und verwalten können. Zu den wichtigsten Funktionen gehören:

•	Aufgaben erstellen und verwalten: Lehrer können neue Aufgaben zur Datenbank hinzufügen und bestehende Aufgaben bearbeiten oder löschen.

•	Aufgabensuche: Die Suchfunktion ermöglicht es den Nutzern, spezifische Aufgaben nach Stichworten oder Kategorien zu finden.

•	Warenkorbsystem: Lehrer können ausgewählte Aufgaben in einem Warenkorb sammeln und daraus Aufgabenblätter erstellen.

Die Anwendung wurde erfolgreich implementiert und erfüllt die gestellten Anforderungen.



