\subsection{Der Server}

Der Server erfüllt im Wesentlichen zwei Hauptaufgaben:
\begin{enumerate}
    \item Authentifizierung des Users: Testen, ob ein User die Berechtigung hat auf bestimmte Teile der Webseite zuzugreifen.
    \item Kommunikation mit der Datenbank: Angfragen vom User weiterleiten, die Antworten verarbeiten und in geeigneter Form an den User zurücksenden.
\end{enumerate}

Die einzelnen Routes werden durch den Express-Router verarbeitet. Somit können diese alle in separaten Dateien gespeichert werden und die Dateien werden übersichtlicher.

\subsubsection{Authentifizierung}
Die Authetifizierung läuft über das package \verb|passport.js|, das die notwendigen Tools für eine Registrierung und Authentifizierung liefert. Die Aufsetzung von \verb|passport.js| erfolgte in unseren Fall über eine sogenannte \verb|LocalStrategy| -- wir verwalten also unsere User selber. Es ist auch möglich bestehende Logins wie zum Beispiel Google oder Facebook in einer Seite auf ähnliche Weise einzubinden. 

Im Code unten ist ein Ausschnitt aus der Konfigurierung zu sehen, wo ein User verifiziert wird: Falls kein User mit dem eingegebenen \verb|username| existiert wird in der Callback Funtion \verb|false| mitgegeben. Falls alles korrekt ist, kommt der User als Objekt in die Callback Funktion. Diesen User kann man dann verwenden, um einen lokalen User zu setzen und ihm Berechtigungen zu geben.

Ebenfalls sehen wir, dass das Passwort nicht im Plaintext verwendet wird, sondern mit dem Package \verb|argon2| 'gehasht' wird. Die Verifizierung läuft dann über einen Vergleich von 'Hashes'. 

 \newpage

\begin{lstlisting}
const localStrategy = new LocalStrategy(
  { usernameField: "username", passwordField: "password" },
  verify
);
/**
 * Verify function for passport
 */
async function verify(username, password, callback) {
  try {
    const { user, success } = await getUser({ username: username });
    //check if user exists
    if (!success) {
      return callback(null, false);
    }

    //user exists check for valid pw
    if (await argon2.verify(user.password, password)) {
      return callback(null, user);
    }
    return callback(null, false);
  } catch (e) {
    return callback(e);
  }
}
\end{lstlisting}

Nachdem der User authentifiziert wurde, kann er dann auf die Funktionalitäten der Applikation zugreifen: Aufgaben Suchen etc. Insbesondere kann ein User auch seine eigenen Aufgaben editieren oder löschen. Die \verb|UserId| wird von \verb|passport.js| in den lokalen Speicher geschrieben und wird gebraucht, um den Editierknopf anzuzeigen. 

\subsubsection{Kommunikation mit der Datenbank}

Der Server hat ebenfalls die Aufgabe mit der Datenbank zu kommunizieren. Hier lautet das Stichwort 'CRUD' -- 'Create, Read, Update, Delete'. Diese sind die Funktionalitäten, die der User in der Interaktion mit der Datenbank haben möchte:
\begin{enumerate}
    \item Aufgaben erstellen (Create)
    \item Aufgaben suchen resp. anschauen (Read)
    \item Eigene\footnote{Sowohl beim editieren, wie auch beim löschen von Aufagben wird die Identität des Users noch einmal auf dem Server verifiziert. Somit kann man nicht durch lokale Anpassung des Users Aufgaben in der Datenbank verändern.} Aufgaben editieren (Update)
    \item Eigene Aufgaben löschen (Delete)
\end{enumerate}

Um ein bisschen Struktur in diese Funktionalitäten zu bringen, sind diese Anfragen als separate Funktionen im Unterordner \verb|controllers| gespeichert. Dies erlaubt es Server Funktionen von Datenbank Funktionen zu trennen. Es handelt sich hier um Funktionen, mit unter Anderem folgenden Aufgaben:
\begin{itemize}
    \item Kategorien aus der Datenbank lesen
    \item Aufgaben in die Datenbank schreiben
    \item Neue User erstellen
    \item etc.
\end{itemize}

Jede dieser Aufgaben entspricht dann einer Funktion, die an geeigneter Stelle aufgerufen werden kann\footnote{Somit bewegen wir uns in die Richtung der funktionellen Programmierung. Wir haben uns sicher nicht konsequent daran festgehalten, aber wo immer günstig, haben wir so gearbeitet. Dies gibt auch wieder eine grössere Übersicht und man kann einfacher Fehler finden.}. 

Da wir für die Datenbank Prisma verwenden, sind diese Anfragen wiederum als eingebaute Funktionen erhältlich und wir müssen einfach die korrekten Funktionen mit den gewünschten Eingaben aufrufen. 

Zum Beispiel liefert der untenstehende Controller die fünf neusten Aufagben aus der Datenbank:

\newpage
\begin{lstlisting}
async function getRecentExercises() {
    try {
        const exercises = await prisma.exercise.findMany({
        take: 5,
        orderBy: {
            updatedAt: "desc",
        },
        });
        return exercises;
    } catch (error) {
        return { msg: "Error in DB request", err: error };
    }
}
\end{lstlisting}


