\section{Vorwort}

Auf der Suche nach einem geeigneten Projekt für die Abschlussarbeit der Gyminf Weiterbildung, war vor allem ein Punkt wichtig: Das Resultat dieser Arbeit sollte etwas sein, was in unserer Unterrichtstätigkeit nützlich ist. 

Die Inspiration eine Aufgaben Datenbank zu erstellen kam von einer -- aktuell nicht mehr aktiven -- Webseite namens \verb|munterbunt.ch|. Auf dieser konnte man aus einer Liste von Aufgaben wählen und diese dann in einem PDF herunterladen. 

Über mehrere Monate haben wir (Oliver De Capitani und Patrick Weber) uns daran gemacht diese Idee umzusetzten. Wir möchten an dieser Stelle besonders folgende Personen danken, die uns dabei unterstützt haben: 

Vielen Dank an Urs Meyer -- unser Betreuer an der FHNW Nordwestschweiz. Er hat immer unser Potenzial gesehen und uns sein Vertrauen geschenkt. 

Zudem möchten wir noch ein grosses Dankeschön an Cedric Geissmann geben, der uns immer wieder mit technischen Lösungen unterstützt hat. Gerade die Authentifizierungs- und Zertifizierungsaspekte dieses Projekts wären ohne seine Hilfe nicht möglich gewesen.

Unsere Applikation hat nun folgende Funktionalitäten.
\begin{itemize}
  \item Registrierung und Login für Benutzer
  \item Aufgaben aus einer gemeinsamen Datenbank lesen, filtern und auswählen
  \item Ausgewählte Aufgaben im \LaTeX\  Format herunterladen
  \item Eigene Aufgaben kreieren, editieren oder löschen
\end{itemize}

Die Webapplikation ist unter der URL \url{https://letstalkaboutx.ch} erreichbar und kann und soll frei genutzt werden.