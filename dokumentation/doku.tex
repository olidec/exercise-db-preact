% Im header stehen die Grundeinstellungen des Dokuments
\documentclass[12pt,a4paper]{scrartcl} % 12 Punkte Schrift, A4 Papier
\usepackage[utf8]{inputenc} 
\usepackage[ngerman]{babel} % Passt alle automatischen Texte an. 
%\usepackage[english]{babel} % Diese Variante verwenden, fall man eine Arbeit auf Englisch schreibt.

\parindent=0cm
\parskip=0.3cm
\linespread{1.5}

\usepackage{graphicx}
\graphicspath{{images/},{../images/}}

\newcommand{\centfig}[2]{\begin{center}
  \includegraphics[width=#1\textwidth]{#2}
  \end{center}}




% Ein paar standard Pakete. Man muss genau wissen, wofür alle gebraucht werden...
\usepackage{amsmath}
\usepackage{amsfonts}
\usepackage{amssymb}
\usepackage{pdfpages}
\usepackage[hidelinks]{hyperref}
\usepackage{datetime}
\usepackage{float}

%Code
\usepackage{listings}
\usepackage{xcolor}


\lstset{
    language=Python,
    basicstyle=\ttfamily\footnotesize,
    keywordstyle=\color{blue},          % Schlüsselwörter (z.B. function, if, else)
    stringstyle=\color{orange},         % Zeichenketten
    commentstyle=\color{gray},          % Kommentare
    numberstyle=\tiny\color{gray},      % Zeilennummern
    stepnumber=1,                       % Zeilennummernschritt
    numbersep=10pt,                     % Abstand zu Zeilennummern
    showspaces=false,                   % Leerzeichen nicht anzeigen
    showstringspaces=false,             % Leerzeichen in Zeichenketten nicht anzeigen
    breaklines=true,                    % Zeilenumbruch bei langen Zeilen
    frame=single,                       % Rahmen um den Code
    tabsize=2,                          % Tab-Breite
    captionpos=b,                       % Position der Beschriftung (b=unten)
    morekeywords={console, log},        % Zusätzliche Schlüsselwörter
    keywordstyle=[2]\color{purple},     % Stil für spezielle Schlüsselwörter (z.B. console, log)
    identifierstyle=\color{teal},       % Stil für Variablen und Funktionsnamen
    numberstyle=\tiny\color{gray},      % Stil für Zeilennummern
    emph={function, return, var},       % Hervorhebung spezifischer Schlüsselwörter
    emphstyle=\color{magenta},          % Stil der hervorgehobenen Wörter
    commentstyle=\color{gray}, % Stil für Kommentare (z.B. grünlich)
    literate=
    *{0}{{{\color{red}0}}}{1}           % Zahlen in Rot
     {1}{{{\color{red}1}}}{1}
     {2}{{{\color{red}2}}}{1}
     {3}{{{\color{red}3}}}{1}
     {4}{{{\color{red}4}}}{1}
     {5}{{{\color{red}5}}}{1}
     {6}{{{\color{red}6}}}{1}
     {7}{{{\color{red}7}}}{1}
     {8}{{{\color{red}8}}}{1}
     {9}{{{\color{red}9}}}{1}
     {=>}{{{\color{blue}=>}}}{2},    
     comment=[l]{//}  % Legt // als Kommentarzeichen fest   % Operatoren hervorheben
}



% Informationen für die Titelseite
\title{Webapplikation zur Erstellung, Sammlung und Ausgabe von Mathematikaufgaben}
\subtitle{Abschlussprojekt im Rahmen der Gyminf Weiterbildung}
\date{\today}
\author{Oliver De Capitani \and Patrick Weber}




% Der Anfang des Dokuments
\begin{document}



\maketitle % Hier wird die Titelseite mit den obigen Informationen eingefügt. Falls man eine "kunstvollere" Titelseite mit einem anderen Programm erstellen möchte, kann man sie hier einfügen. Dafür muss man die Titelseite im gleichen Verzeichnis (z.B. mit dem Namen TitelseiteMA2016.pdf) im pdf Format ablegen und mit dem Befehl
% \includepdf{TitelseiteMA2016}
% wird sie dann eingefügt.



\newpage % Eine neue Seite wird begonnen...
\tableofcontents % Hier wird automatisch das Inhaltsverzeichnis eingefügt. Achtung: Änderungen werden erst nach dem zweiten kompilieren sichtbar.

\newpage
\section{Vorwort}

Auf der Suche nach einem geeigneten Projekt für die Abschlussarbeit der Gyminf Weiterbildung, war vor allem ein Punkt wichtig: Das Resultat dieser Arbeit sollte etwas sein, was in unserer Unterrichtstätigkeit nützlich ist. 

Die Inspiration eine Aufgaben Datenbank zu erstellen kam von einer -- aktuell nicht mehr aktiven -- Webseite namens \verb|munterbunt.ch|. Auf dieser konnte man aus einer Liste von Aufgaben wählen und diese dann in einem PDF herunterladen. 

Über mehrere Monate haben wir (Oliver De Capitani und Patrick Weber) uns daran gemacht diese Idee umzusetzten. Wir möchten an dieser Stelle besonders folgende Personen danken, die uns dabei unterstützt haben: 

Vielen Dank an Urs Meyer -- unser Betreuer an der FHNW Nordwestschweiz. Er hat immer unser Potenzial gesehen und uns sein Vertrauen geschenkt. 

Zudem möchten wir noch ein grosses Dankeschön an Cedric Geissmann geben, der uns immer wieder mit technischen Lösungen unterstützt hat. Gerade die Authentifizierungs- und Zertifizierungsaspekte dieses Projekts wären ohne seine Hilfe nicht möglich gewesen.

Unsere Applikation hat nun folgende Funktionalitäten.
\begin{itemize}
  \item Registrierung und Login für Benutzer
  \item Aufgaben aus einer gemeinsamen Datenbank lesen, filtern und auswählen
  \item Ausgewählte Aufgaben im \LaTeX\  Format herunterladen
  \item Eigene Aufgaben kreieren, editieren oder löschen
\end{itemize}

Die Webapplikation ist unter der URL \url{https://letstalkaboutx.ch} erreichbar und kann und soll frei genutzt werden.

\newpage

 
\section{Einleitung}

Diese Arbeit hatte zum Ziel eine Webapplikation zu kreieren, auf der man Aufgaben kreieren, aussuchen und anschliessend herunterladen kann. Wie man sich vorstellen kann ist dies ein grosses Projekt mit Potenzial zum endlosen Ausbau. Aus diesem Grund haben wir uns in unserer Vereinbarung auf folgende Grundfunktionalitäten beschränkt\footnote{"beschränkt" war immer noch komplex genug.}:
\begin{itemize}
  \item Suchen und Filtern von Aufgaben aus der Datenbank, die in einer Vorschau (z.B. via KaTeX) in der Applikation dargestellt werden
  \item Speichern von gewünschten Aufgaben in einer Art Warenkorb, von wo aus sie anschliessend im PDF oder im LaTeX Format heruntergeladen werden können
  \item Login Maske, damit die Aufgaben nur registrierten Benutzern (Lehrpersonen) zugänglich sind
  \item Benutzerkonto Administration (Passwort ändern, Konto löschen etc.)
  \item Formular zum Erstellen von eigenen Aufgaben
\end{itemize}

Im Grossen und Ganzen ist es uns gelungen diese Funktionalitäten zur Verfügung zu stellen. Einzig die Administration des Benutzerkontos steht -- in der vorliegenden Fassung -- nicht zu Verfügung. Aufgrund des drohenden "feature creeps" wurde beschlossen, auf diese momentan zu verzichten. Personenadministration kann -- auf Anfrage -- von einem berechtigten Benutzer über den server direkt erledigt werden.

Das Arbeiten an diesem Projekt erforderte viel Koordination und das Erlernen von neuen Ideen und Konzepten.  Durch dieses Projekt sind wir einer Vielzahl von Techniken und Begriffen begegnet: Server, Client, Datenbank, Full-Stack, Express, JavaScript, React, Preact, Docker, \LaTeX, Authentifizierung, Zertifikate etc. Einige kennen wir nun sehr gut, andere einfach gut genug, um damit zu arbeiten. Es war auf jeden Fall nicht möglich in all diesen Bereichen bis zum Expertentum vorzudringen. Weil wir aber auf so vielen verschiedenen Ebenen gearbeitet haben, haben wir einen einzigartigen Einblick in die komplette Bau- und Funktionsweise einer Webapplikation erhalten. Diese Erfahrungen und das erlernte Wissen, wird uns sicher in unserer Arbeit als Informatik-Lehrpersonen helfen.

Das Endprodukt dieser Arbeit ist eine full-stack Webapplikation, die auf einem Linode Server läuft. Die Applikation läuft in einem Docker Container und ist durch ein selbst-gezeichnetes Zertifikat gesichert.

Die Applikation kann unter \url{https://letstalkaboutx.ch} aufgerufen werden.

 \newpage

 \section{Struktur der Webapplikation}

In der folgenden Graphik sehen wir schematisch die Struktur unserer Webapplikation: Das gesamte läuft in einem Docker Container. Ein Proxy (ein zwischengeschalteter Server) vermittelt zwischen dem User und den verschiedenen Bereichen des Docker Containers. Anfragen werden an den Client gesendet.



\subsection{Client: Frontend-Architektur:} 

Im Frontend haben wir die verschiedenen Komponenten der Anwendung in eine Hierarchie gebracht, damit Änderungen an einer Komponente nicht ungewollt andere Teile der Applikation beeinflussen. 
Es handelt sich dabei um folgende Komponenten:

\begin{itemize}
  \item das Suchen der Aufgaben (AufgabenSuchen-Komponente)
  
  \item die detaillierten Ansicht einer Aufgabe (AufgabeDetails-Komponente)
\item das Anzeigen des Warenkorbsystems (Warenkorb-Komponente) 
  
  \item das Hinzufügen von Aufgaben (AufgabeHinzufügen-Komponente)
  
  \item das Editieren von Aufgaben (AufgabeEditieren-Komponente)
  

\end{itemize}

Diese Komponentenhierarchie führt zu einem besseren Überblick über die verschiedenen Funktionen und die Applikation wird in eine saubere und wartbare Codebasis gebracht.


\subsubsection{Die Komponentenhierarchie}

\texttt{3.4.1.1 Die AufgabenSuchen-Komponente:}
\begin{figure}[ht]
  \centfig{0.45}{Suchen_Komp.png}
  \caption{Die Aufgabensuchen-Komponente \cite{fig:aufgabensuche}}
\end{figure}


\begin{itemize}

\item Aufgabe (Card):
Grundbaustein einer Aufgabe, die den Inhalt, die Lösung, der Schwierigkeitsgrad, den Autor, die Kategorie und die Unterkategorie enthält.

\item Suchergebniss (SearchCard):
Komponente für die Anzeige eines Suchergebnisses, die eine einzelne Aufgabe (Card) darstellt, mit der Möglichkeit die Aufgabe zum Warenkorb hinzuzufügen bzw. aus dem Warenkorb zu löschen, die Details der Aufgabe (AufgabeDetails) anzeigen zu lassen und, sofern man Autor der Aufgabe ist, die Aufgabe zu editieren.

\item Auflistung aller Sucherergebnisse (SearchKorb):
Die einzelnen Aufgaben (SearchCard) werden in der Komponente SearchKorb zusammen angezeigt.
Diese Aufgaben kann man nach Sprache und Schwierigkeitsgrad filtern.

\item AufgabenSuchen (Seitenkomponente):
Diese Seitenkomponente umfasst zwei Komponenten (FindExSubCat und FindExBySearchText), die unterschiedliche Methoden zur Aufgabensuche anbieten. 

Mit FindExSubCat können Aufgaben nach Kategorien und Unterkategorien über eine API Anfrage \texttt{(askServer(`/api/ex?cat ={selectedCategory}\&subcat=\\{subcategoryName}`,"GET"))} gesucht werden (siehe Kapitel 3.4.5 Zustandsmanagement für die Komponente FindExSubCat).

Mit FindExBySearchText kann eine Aufgabe nach einer textbasierten Suchanfrage (inputValue) gesucht werde, mit der API Anfrage \texttt{askServer(`/api/ex?search=\\\${inputValue}`,"GET")}.

Beides Mal werden die anzuzeigenden Aufgaben in der Komponente SearchKorb angezeigt.


\end{itemize}

\texttt{3.4.1.2 Die AufgabeDetails-Komponente:}

\begin{figure}[H]
  \centfig{0.35}{AufgDetails.png}
  \caption{Die Aufgabendetails \cite{fig:aufgabendetails}}
\end{figure}

\begin{itemize}
\item Details einer Aufgabe (AufgabeDetails):
Eine Detailansicht für jede Aufgabe erfolgt in einer Modal Ansicht (Fenster innerhalb eines Browser Fensters erscheint), wobei neben dem Aufgabentext auch die Lösung, die Kategoriezugehörigkeit, der Schwierigkeitsgrad, die Sprache und der Autor der Aufgabe zu sehen ist. 
Die API Anfrage \texttt{askServer("/api/ex?id=\${id}", "GET")} ruft von der Datenbank die Aufgabe mit der entsprechenden ID ab.
Die Aufgabe kann auch in dieser detaillierten Ansicht zum Warenkorb hinzugefügt werden oder aus dem Warenkorb gelöscht werden.



\end{itemize}

\texttt{3.4.1.3 Die Warenkorb-Komponente:}


\begin{figure}[H]
  \centfig{0.35}{Warenkorb.png}
  \caption{Der Warenkorb \cite{fig:warenkorb}}
\end{figure}

\begin{itemize}

\item Aufgabe im Warenkorb (WarenCard):
Eine Aufgabe (Card) im Warenkorb kann gelöschte werden und ihre Position mit der vorhergehenden oder nachfolgenden Aufgabe vertauscht werden, um die richtige Reihenfolge für das herunterladen der Aufgabe zu erhalten. 


\item Warenkorb (Seitenkomponente):
Die einzelnen Aufgaben (WarenCard) werden untereinander im Warenkorb angezeigt und über einen Button können alle Aufgabe im Warenkorb heruntergeladen werden.


\end{itemize}

\texttt{3.4.1.4 Die AufgabeHinzufügen-Komponente:}
\begin{figure}[H]
  \centfig{0.2}{ExForm.png}
  \caption{Aufgaben Hinzufügen \cite{fig:hinzufuegen}}
\end{figure}

\begin{itemize}

\item{Formularfelder einer Aufgabe (ExerciseForm)}:
Diese Komponente umfasst Formularfelder für die Eingabe von Sprache, Schwierigkeitsgrad, Kategorie, Unterkategorie, Aufgabentext und Lösung.

\item ExForm (Seitenkomponente): :
Diese Komponente ermöglicht das Erstellen neuer Aufgaben. Über eine API-Anfrage \texttt{(askServer(/api/ex, "POST", ex))} werden die Daten an das Backend gesendet und die Aufgabe wird, sofern alle Formularfelder richtig ausgefüllt sind, in die Datenbank hinzugefügt.


\end{itemize}

\texttt{3.4.1.5 Die AufgabeEditieren-Komponente:}
\begin{figure}[H]
  \centfig{0.2}{EditForm.png}
  \caption{Aufgaben Editieren \cite{fig:editieren}}
\end{figure}

\begin{itemize}
\item{EditForm} (Seitenkomponente):
Die Sprache, der Schwierigkeitsgrad, die Kategorie, die Unterkategorie, der Aufgabentext und die Lösungen können über Formularfelder aktualiert werden, sofern man Autor der Aufgabe ist.
Die API Anfrage \texttt{askServer("/api/ex", "PUT", exWithCategory)} aktualisiert die Aufgabe in der Datenbank.


\end{itemize}
\subsubsection{Weiteren Seitenkomponenten}

\begin{itemize}
\item Startseite:
 Die Startseite (Landingpage) zeigt eine zusammenfassende und einladende Beschreibung der Webseite an. Über einen Registeren/Login-Button gelangt man zur Loginseite. Hat man noch keine Login, kann man über einen Link zur Registrierungsseite wechseln.

\item Login-Komponente:
Diese Login-Komponente ermöglicht es dem Benutzer, seinen Benutzernamen und sein Passwort in ein Formular einzugeben. Die API Anfrage \texttt{(askServer("/login", "POST", {username,password}))} sendet die Daten an das Backend.
Beim Absenden des Formulars wird ein Cookie mit den Authentifizierungsdaten gesendet. Dieses Cookie wird in der Anwendung gespeichert und geprüft, um den Zugang zu geschützten Bereichen der Anwendung zu gewähren.


\item Registrieren-Komponente:
Die Registrieren-Komponente ermöglicht es dem Benutzer über die Email Adresse, den Benutzernamen und sein Passwort, einen neuen Benutzer anzulegen. Die API Anfrage \texttt{(askServer("/register", "POST", user))} sendet die Daten an das Backend.

\item Navigations-Komponente:
Die Menu-Komponente stellt die Hauptnavigation der Anwendung bereit und ermöglicht den Zugriff auf die verschiedenen Seiten und Funktionen der App, darunter die Startseite (/), das Hinzufügen von Aufgaben (/add), das Finden von Aufgaben (/find), den Warenkorb (/warenkorb), Infos zur Webseite (/about), die Registrierung(/register) und der Login (/login).


\end{itemize}



\subsubsection{Zustandsmanagement mit useContext } 

Ein effektives Zustandsmanagement ist entscheidend für die Verwaltung der Anwendungsdaten sowie der verschiedenen UI-Zustände, wie Benutzereingaben oder geladene Daten (gesuchte Aufgaben). Preact bietet uns hierfür mehrere Optionen: Für lokale Zustände innerhalb einzelner Komponenten nutzen wir die useState und useEffect Hooks sowie Signals (Kapitel 3.4.4), die eine einfache und reaktive Verwaltung dieser Zustände ermöglichen. 

Für komplexe Szenarien, in denen ein globaler Zustand erforderlich ist – also Daten, die von mehreren Komponenten und deren Unterkomponenten (Children) gleichzeitig genutzt werden – greifen wir auf die useContext API zurück. Mit \texttt{useContext} können wir eine zentrale Datenquelle erstellen, die es verschiedenen Komponenten ermöglicht, auf dieselben Daten oder Funktionen zuzugreifen und diese zu aktualisieren, was die Anwendung sowohl leistungsfähig als auch flexibel macht.

Insgesammt wurde zwei zentrale Datenquellen (Kontext-Provider), SearchContext und WarenkorbContext, erstellt. 

\texttt{3.4.3.1 Der SearchContext}

Der SearchContext stellt sicher, dass wichtige Zustandsinformationen wie Suchergebnisse, ausgewählte Kategorien und Benachrichtigungen zentral gespeichert und zwischen verschiedenen Komponenten geteilt werden können. 


\begin{lstlisting}[language=Python]
const { showNotification, cartSearch, setCartSearch, searchText, categor, deleteCard } = useContext(SearchContext);
 \end{lstlisting}  
 
Hier werden die Variablen und Funktion (\texttt{showNotification, setCartSearch, cartSearch, searchText, categor und deleteCard}) aus dem SearchContext abgerufen, die als Signals (Kapitel 3.4.4) implementiert sind und im localStorage des Browsers gespeichert werden, da diese Daten zwischen verschiedenen Komponenten geteilt werden müssen.

\begin{itemize}
  \item \texttt{showNotification} speichert und verwaltet Benachrichtigungen oder Fehlermeldungen, die beim Bearbeiten, Eingeben oder Suchen von Aufgaben auftreten und zeigt sie in roter oder grüner Farbe für eine kurze Zeit an.
  \item \texttt{cartSearch} speichert die Suchergebnisse in einem Array und verwaltet diese.
    \item \texttt{setCartSearch} speichert die Suchergebnisse in der Variablen cartSearch, die von der Benutzerabfrage zurückgegeben werden.
    \item \texttt{searchText} enthält den eingegebenen Suchtext nach dem der Benutzer die Aufgaben sucht und zeigt ihn oberhalb der Suchergebnisse an.
    \item \texttt{categor} speichert die vom Benutzer angeklickte Kategorie und Subkategorie und zeigt sie oberhalb der Suchergebnisse an.
    \item \texttt{deleteCard} entfernt eine Aufgabe aus der Datenbank, sofern man Autor der Aufgabe ist.
  
\end{itemize}


Folgende Grafik zeigt, an welche Komponenten (und immer auch alle untergeordneten Komponenten (Children)) der SearchContext die verschiedene Variablen und Funktionen weitergibt:
\begin{figure}[H]
\centfig{0.3}{all.png}
\caption{Komponenten mit Zugriff auf den SearchContext \cite{fig:all}}
\end{figure}


\texttt{3.4.3.2 Der WarenkorbContext}

Der WarenkorbContext stellt sicher, dass Zustandinsformationen zwischen dem Warenkorb und dem Anzeigen der gesuchten Aufgaben zentral gespeichert und zwischen Komponenten geteilt werden können.

\begin{lstlisting}[language=Python]
const { cartItems, addToKorb, handleDelete, getIndex, getCartCount } = useContext(WarenkorbContext);
 \end{lstlisting} 

 \begin{itemize}

  \item \texttt{addToKorb} speichert die Aufgabe in der Warenkorb-Liste.
  \item \texttt{handleDelete} entfernt eine Aufgabe aus der Warenkorb-Liste.
  \item  \texttt{cartItems} speichert die ausgewählten Aufgaben für dem Warenkorb in der Warenkorb-Liste und verwaltet diese.
  \item \texttt{getIndex} gibt die Position einer Aufgabe in der Warenkorb-Liste an. (Ist der Index -1, ist die Aufgabe nicht im Warenkorb und der Button addToKorb wird angezeigt.)
  \item \texttt{getCartCount} gibt die Anzahl der gespeicherten Aufgaben in der Warenkorb-Liste an.
 \end{itemize}

 Folgende Grafik zeigt, an welche Komponenten (und immer auch alle untergeordneten Komponenten (Children)) der WarenkorbContext die verschiedene Variablen und Funktionen weitergibt:
 \begin{figure}[ht]
 \centfig{0.3}{all2.png}
 \caption{Komponenten mit Zugriff auf den WarenkorbContext \cite{fig:all2}}
 \end{figure}


 \subsubsection{Zustandsmanagement mit useState, useEffect und Signals}

 Der useState Hook ermöglicht es, Zustände innerhalb einer Funktionskomponente zu verwalten. Wenn man useState aufruft, erhält man ein Array mit zwei Werten: den aktuellen Zustand und eine Funktion, um diesen Zustand zu aktualisieren.
 
 \begin{lstlisting}
 const [state, setState] = useState(initialValue);
 // state: Die aktuelle Zustandsvariable, den man verwenden kann.
 // setState: Eine Funktion, mit der man den Zustand der Variable aktualisieren kann.
 //initialValue: Der Anfangswert des Zustands.
 
 \end{lstlisting}
 
 
 Der useEffect Hook wird verwendet, um Seiteneffekte (dependencys) in Funktionskomponenten zu behandeln. Ein Seiteneffekt ist z.B. das Abrufen von Daten von einem Server oder das Ändern des DOM. Sofern es eine Änderung der Variablen (dependencys) in diesem Array gibt, wird die useEffect()-Funktion erneut ausgeführt.
 
 \begin{lstlisting}
 useEffect(() => {
  // Effekt-Funktion, z.B. Daten laden oder DOM aktualisieren
   return () => {
     // Cleanup-Funktion, z.B. Event-Listener entfernen
   };
 }, [dependency1, dependency2]);
 
 \end{lstlisting}
   
 
 
Zusätzlich bietet Preact die Möglichkeit, Signals zu verwenden, die eine noch feinere Kontrolle über den reaktiven Datenfluss bieten.
Signals sind reaktive Zustandscontainer, die es ermöglichen, Zustandsänderungen automatisch zu verfolgen und so Zustände innerhalb der Komponente und zwischen den Komponenten zu aktualisieren. Komponenten, die ein Signal lesen, werden automatisch neu gerendert, wenn sich das Signal ändert.
 

\subsubsection{Code-Beispiel: Zustandsmanagement für die Komponente FindExSubCat } 
Das folgende Beispiel beziehen sich auf die Komponente \texttt{FindExSubCat} um Aufgaben nach Kategorien und Subkategorien durch clicken auf dieselben zu suchen.

Das Signal \texttt{cat} wird verwendet um alle Kategorien vom Server zu laden.

\begin{lstlisting}
const cat = signal([]);
const loadCat = async () => {
  const res = await askServer("/api/cat/", "GET");
  cat.value = res.response;
};

\end{lstlisting}

Nun wird die useEffect()-Funktion verwendet damit beim laden der Seite die loadCat()-Funktion die Kategorien vom Server lädt und sie in das Array categories schreibt \\\texttt{(setCategories(cat.value))}. Da sich diese Variable beim laden der Seite verändern könnte, wird sie als useState Variable definiert. 
Diese useEffect()-Funktion wird nur beim laden der Seite ausgeführt, da die Liste der Abhängigkeiten (dependencys, leeres Array auf der letzten Zeile) leer bleibt. 

\begin{lstlisting}
import { cat, loadCat } from "../signals/categories.js";
const [categories, setCategories] = useState([]);
useEffect(() => {
    const fetchCategories = async () => {
    await loadCat();
    setCategories(cat.value);
    };
    fetchCategories();
  }, []);

\end{lstlisting}


Die Funktion \texttt{onCategoryClick} wird aufgerufen, wenn eine Kategorie angeklickt wird. Sie setzt die ausgewählte Kategorie in die Variable \texttt{selectedCategory} und setzt das Array der Unterkategorie auf einen leeren String zurück. Dann sendet die Funktion eine Anfrage an den Server, um Übungen basierend auf der Kategorie zu laden und in der Variablen \texttt{excat} zu speichern.

\begin{lstlisting}[language=Python]
const [selectedCategory, setSelectedCategory] = useState("");
const [selectedSubcategory, setSelectedSubcategory] = useState("");

const onCategoryClick = async (categoryName) => {
setSelectedCategory(categoryName);
setSelectedSubcategory(""); # Reset subcategory when selecting a new category
const route = `/api/ex?cat=${categoryName}`;
const res = await askServer(route, "GET");
const excat = res.response;
    
\end{lstlisting}

 Wenn der Server keine Antwort auf die Anfrage gibt (Status ungleich 200) oder keine Übungen zurückgibt, wird eine Benachrichtigung (showNotification) angezeigt und das Array der gesuchten Aufgaben sowie die Kategorieauswahl zurückgesetzt. Andernfalls wird die Liste der gefundenen Übungen mit der Funktion \texttt{setCartSearch} in die Variablen \texttt{cartSearch} gespeichert. (siehe Kapitel 3.4.3.1 SearchContext)


\begin{lstlisting}[language=Python]
    if (res.status != 200 || excat.length === 0) {
      showNotification("No exercise matches the search term.", "red");
      setCartSearch([]);
      setSelectedCategory("");
      searchText.value = "";
      categor.value[0] = categoryName;
      categor.value[1] = "";
    } else {
      setCartSearch(excat);
      searchText.value = "";
      categor.value[0] = categoryName;
      categor.value[1] = "";
    }
  };


\end{lstlisting}



Die useEffect() Funktion für Unterkategorien sucht die Unterkategorien der ausgewählten Kategorie, die in der Variablen \texttt{selectedCategory} gespeichert wurde und speichert diese im Array \texttt{subcategories} und zwar jedes Mal, wenn sich \texttt{selectedCategory} oder \texttt{categories} ändert (Seiteneffekt). 


\begin{lstlisting}[language=Python]
const [subcategories, setSubcategories] = useState([]);
useEffect(() => {
    if (selectedCategory) {
      const category = categories.find((c) => c.name === selectedCategory);
      setSubcategories(category ? category.subcategories : []);
    } else {
      setSubcategories([]);
    }
  }, [selectedCategory, categories]);
      
\end{lstlisting}



Die Funktion \texttt{onSubcategoryClick} wird aufgerufen, wenn eine Unterkategorie ausgewählt wird. Sie sendet eine Anfrage an den Server, um Übungen basierend auf der ausgewählten Unterkategorie zu laden und und in der Variablen \texttt{exsubcat} zu speichern.

\begin{lstlisting}[language=Python]
const onSubcategoryClick = async (subcategoryName) => {
    setSelectedSubcategory(subcategoryName);

    const route = `/api/ex?cat=${selectedCategory}&subcat=${subcategoryName}`;
    const res = await askServer(route, "GET");
    const exsubcat = res.response;
    
    
\end{lstlisting}

 Ähnlich wie bei der Kategorieauswahl wird eine Benachrichtigung angezeigt und die Unterkategorieauswahl zurückgesetzt, wenn keine passenden Übungen gefunden werden. Ansonsten wird die Liste der Übungen mit der Funktion \texttt{setCartSearch} gespeichert. (siehe Kapitel 3.4.3.1 SearchContext).


\begin{lstlisting}[language=Python]
if (res.status != 200 || exsubcat.length === 0) {
      showNotification("No exercise matches the search term.", "red");
      setCartSearch([]);
      searchText.value = "";
      categor.value[1] = subcategoryName;
    }
else {
      setCartSearch(exsubcat);
      searchText.value = "";
      categor.value[1] = subcategoryName;
    }
  };
\end{lstlisting}


\subsection{Datenbank}

\subsubsection{Datenbankstruktur}

In diesem Abschnitt wird das Datenbankmodell beschrieben, welches für die Speicherung der Benutzerinformationen, Aufgaben und deren Kategorisierung verwendet wird. Die Implementierung erfolgt mit Prisma ORM und PostgreSQL als Datenbank. Im Folgenden wird die Struktur der relevanten Tabellen dokumentiert.

Das Datenbankmodell umfasst vier Haupttabellen: \texttt{User}, \texttt{Exercise}, \texttt{Category} und \texttt{Subcategory}. Diese Tabellen sind miteinander verknüpft, sodass Benutzer Aufgaben erstellen und diese in Kategorien und Unterkategorien organisieren können.

\newpage
\begin{lstlisting}[language=Python]
  generator client {
  provider        = "prisma-client-js"
  previewFeatures = ["fullTextIndex", "fullTextSearch"]
}

datasource db {
  provider = "postgresql"
  url      = env("DATABASE_URL")
}

model User {
  id        Int        @id @default(autoincrement())
  email     String     @unique
  username  String     @unique 
  password  String?
  role      String     @default("USER")
  exercises Exercise[]
  retry     Int        @default(0)
  retryExp  DateTime?  
}

model Exercise {
  id            Int         @id @default(autoincrement())
  createdAt     DateTime    @default(now())
  updatedAt     DateTime    @updatedAt
  summary       String? // Optional, kann null sein
  content       String
  solution      String
  language      String      @default("Deutsch")
  difficulty    Int         @default(1)
  authorId      Int         @default(1)
  author        User        @relation(fields: [authorId], references: [id])
  categoryId    Int         @default(1)
  categories    Category    @relation(fields: [categoryId], references: [id])
  subcategoryId Int @default(1)
  subcategories Subcategory @relation(fields: [subcategoryId], references: [id])
}

model Category {
  id            Int           @id @default(autoincrement())
  name          String
  subcategories Subcategory[]
  exercises     Exercise[]
}

model Subcategory {
  id         Int        @id @default(autoincrement())
  name       String
  categoryId Int
  category   Category   @relation(fields: [categoryId], references: [id], onDelete: Cascade)
  exercises  Exercise[]
}

\end{lstlisting}




\subsubsection{Tabelle \texttt{User}}

Die Tabelle \texttt{User} speichert die Informationen der Benutzer. Jeder Benutzer hat eine eindeutige ID sowie eine E-Mail-Adresse und einen Benutzernamen. Optional kann ein Passwort gespeichert werden. Benutzer können eine Rolle besitzen (standardmäßig \texttt{USER}), die ihre Berechtigungen festlegt. Zusätzlich verwaltet die Tabelle die Aufgaben, die ein Benutzer erstellt hat.

\begin{itemize}
  \item \texttt{id}: Primärschlüssel, automatisch inkrementiert.
  \item \texttt{email}: Eindeutige E-Mail-Adresse des Benutzers.
  \item \texttt{username}: Eindeutiger Benutzername.
  \item \texttt{password}: Optionales Passwortfeld, kann null sein.
  \item \texttt{role}: Rolle des Benutzers (standardmäßig \texttt{USER}).
  \item \texttt{exercises}: Beziehung zur Tabelle \texttt{Exercise}. Verknüpfung zu den von diesem Benutzer erstellten Aufgaben, dies ist eine \texttt{1:n}-Beziehung da der Benutzer mehrer Aufgaben erstellen kann
  \item \texttt{retry}: Anzahl der fehlgeschlagenen Anmeldeversuche.
  \item \texttt{retryExp}: Zeitpunkt, ab dem der Benutzer nach mehreren Fehlversuchen erneut versuchen kann, sich anzumelden.
\end{itemize}

\subsubsection{Tabelle \texttt{Exercise}}

Die Tabelle \texttt{Exercise} speichert die einzelnen Mathematikaufgaben, die von Benutzern erstellt werden. Jede Aufgabe enthält Informationen wie den Inhalt, die Lösung, die Sprache und den Schwierigkeitsgrad. Jede Aufgabe ist einem Benutzer (dem Autor), einer Kategorie und einer Unterkategorie zugeordnet.

\begin{itemize}
  \item \texttt{id}: Primärschlüssel, automatisch inkrementiert.
  \item \texttt{createdAt}: Erstellungsdatum der Aufgabe, standardmäßig die aktuelle Zeit.
  \item \texttt{updatedAt}: Automatisch aktualisiertes Datum, wenn die Aufgabe geändert wird.
  \item \texttt{summary}: Eine optionale Zusammenfassung der Aufgabe.
  \item \texttt{content}: Der eigentliche Inhalt der Aufgabe.
  \item \texttt{solution}: Die Lösung der Aufgabe.
  \item \texttt{language}: Sprache der Aufgabe (standardmäßig \texttt{Deutsch}).
  \item \texttt{difficulty}: Schwierigkeitsgrad der Aufgabe (standardmäßig 1).
  \item \texttt{authorId}: Fremdschlüssel, der den Benutzer referenziert, der die Aufgabe erstellt hat.
   \item \texttt{author}: Beziehung zur Tabelle \texttt{User}, der die Aufgabe erstellt hat. Diese Beziehung ist eine \texttt{1:n}-Beziehung, da ein Benutzer mehrere Aufgaben erstellen kann.
  \item \texttt{categoryId}: Fremdschlüssel, der die Kategorie referenziert, zu der die Aufgabe gehört.
  
\item \texttt{categories}: Beziehung zur Tabelle \texttt{Category}. Diese Beziehung ist eine \texttt{n:1}-Beziehung, da jede Aufgabe genau einer Kategorie zugeordnet ist, aber eine Kategorie mehrere Aufgaben enthalten kann.
  \item \texttt{subcategoryId}: Fremdschlüssel, der die Unterkategorie referenziert, zu der die Aufgabe gehört.
    \item \texttt{subcategories}: Beziehung zur Tabelle \texttt{Subcategory}. Diese Beziehung ist ebenfalls eine \texttt{n:1}-Beziehung, da jede Aufgabe genau einer Unterkategorie zugeordnet ist, aber eine Unterkategorie mehrere Aufgaben umfassen kann.
\end{itemize}

\subsubsection{Tabelle \texttt{Category}}

Die Tabelle \texttt{Category} enthält Informationen über die Hauptkategorien, denen Aufgaben zugeordnet werden können. Eine Kategorie kann mehrere Unterkategorien und Aufgaben beinhalten.

\begin{itemize}
  \item \texttt{id}: Primärschlüssel, automatisch inkrementiert.
  \item \texttt{name}: Name der Kategorie.
   \item \texttt{subcategories}: Beziehung zur Tabelle \texttt{Subcategory}. Diese Beziehung ist eine \texttt{1:n}-Beziehung, da eine Kategorie mehrere Unterkategorien enthalten kann, aber jede Unterkategorie nur zu einer Kategorie gehört.
  \item \texttt{exercises}: Beziehung zur Tabelle \texttt{Exercise}. Diese Beziehung ist ebenfalls eine \texttt{1:n}-Beziehung, da eine Kategorie mehrere Aufgaben enthalten kann, aber jede Aufgabe nur zu einer Kategorie gehört.
\end{itemize}

\subsubsection{Tabelle \texttt{Subcategory}}

Die Tabelle \texttt{Subcategory} speichert Informationen über die Unterkategorien, die einer übergeordneten Kategorie zugeordnet sind. Jede Unterkategorie kann mehrere Aufgaben enthalten.

\begin{itemize}
  \item \texttt{id}: Primärschlüssel, automatisch inkrementiert.
  \item \texttt{name}: Name der Unterkategorie.
  \item \texttt{categoryId}: Fremdschlüssel, der die übergeordnete Kategorie referenziert.
 \item \texttt{category}: Beziehung zur Tabelle \texttt{Category}. Diese Beziehung ist eine \texttt{n:1}-Beziehung, da jede Unterkategorie zu genau einer Kategorie gehört, aber eine Kategorie mehrere Unterkategorien enthalten kann.Beziehung zur übergeordneten Kategorie, bei deren Löschung die Unterkategorie ebenfalls gelöscht wird (Cascade-Löschung).
 
   \item \texttt{exercises}: Beziehung zur Tabelle \texttt{Exercise}. Diese Beziehung ist eine \texttt{1:n}-Beziehung, da eine Unterkategorie mehrere Aufgaben enthalten kann, aber jede Aufgabe nur zu einer Unterkategorie gehört.
\end{itemize}

\subsubsection{Zusammenfassung}

Das beschriebene Datenbankmodell ermöglicht eine klare und logische Strukturierung von Benutzern, Aufgaben, Kategorien und Unterkategorien. Mithilfe von \texttt{Prisma} ORM und \texttt{PostgreSQL} als Datenbank wird sichergestellt, dass die Anwendung skalierbar und effizient auf große Datenmengen zugreifen kann. Jede Tabelle ist durch eindeutige Beziehungen verknüpft, was eine einfache Verwaltung der Daten und eine effiziente Abfrage ermöglicht.


Die Datenbankstruktur unterstützt umfassende Interaktionen zwischen Benutzern, ihren Aufgaben sowie die Kategorisierung von Aufgaben.




\newpage

\section{Prozess}
Die Zusammenarbeit und Koordination war ein wichtiger Teil der Arbeit. Glücklicherweise konnten wir uns relativ gut absprechen und Teile der Applikation unabhängig von einander entwickeln. 

\subsection{Koordination}

In einer ersten Phase haben wir Trello benutzt, um einen Überblick über die notwendigen Komponenten und Teilschritte zu bekommen. Trello ist ein visuelles Tool, um den Arbeitsprozess eines Projekts zu planen und zu koordinieren. Gerade zu Beginn sah man oft vor lauter Bäumen den Wald nicht -- da war ein solches Tool wirklich hilfreich. Das untenstehende Bild zeigt exemplarisch einen Zustand, wie das Tool eingesetzt wurde. Aktive Tasks werden von links nach rechts verschoben, sobald sie bearbeitet oder erledigt werden.
\begin{figure}[ht]
    \centfig{0.6}{trello.png}
    \caption{Screenshot Trello \cite{fig:trello}}
\end{figure}

Mit der Zeit bekamen wir eine bessere Übersicht über die Struktur und notwendigen Komponenten der Webapplikation und wir sahen uns nicht mehr so gezwungen, das Tool zu verwenden. Als eine Zweiergruppe, brauchte es auch immer nur einen einzigen Anruf oder Kurznachricht, um das ganze Team auf denselben Wissensstand zu bringen.

\subsection{Probleme und Lösungen}

Bei einem Projekt dieses Ausmasses, war es uns unmöglich im Voraus alle möglichen Arten von Problemen vorherzusehen. Neben 'normalen' Programmierproblemen, gab es auch einen Haufen von anderen Problemen und Fehlermeldungen, die manchmal sehr kreative Lösungen erforderten (oder solche, die im Nachhinein sich als richtig dumm entpuppten). Wir werden an dieser Stelle keine komplette Aufzählung liefern, aber ein kleines 'best of' präsentieren.

\subsubsection{CORS} 
Ein immer wiederkehrendes Problem war, dass Chrome, Firefox und Co. Mühe damit hatten, dass Anfragen von der Webseite an einen anderen (in der Produktionsphase ungesicherten) Server gingen. Dies führte zu sogenannten \emph{cross-origin resource sharing-} oder CORS-Problemen. Kurz gesagt, bedeutet dies, dass der Browser nicht will, dass Informationen von einer anderen Domain übertragen werden.

Die Lösung war, dass man im Server-Code explizit die Webseite angeben musste, die darauf zugreifen darf. Da in der Schlussversion die einzelnen Teile innerhalb eines Docker Containers waren, hat sich dieses Problem, dann erledigt. Die Docker Konfiguration übernahm dann diese Verbindungen.

\subsubsection{Aufsetzen der Testdatenbank}
Für eine lange Zeit hatten wir Mühe die Testdatenbank auf verschiedenen Geräten zum laufen zu bringen. Das erstaunliche war, dass sie auf einem Gerät funktionierte, aber auf anderen nicht, obwohl eigentlich derselbe Code lief. Es stellte sich heraus, dass das Problem daran lag, dass beim Aufsetzen der Datenbank die Aufgaben IDs manuell gesetzt wurden, aber dass dann der interne Zähler nicht auf dem selben Stand war. Dies führte dazu, dass die Datenbank einen neuen Eintrag mit einer bereits bestehenden ID erstellen wollte und dann abstürzte. Nachdem wir die IDs aus dem Aufsetzcode der Datenbank gelöscht hatten, funktionierte es wieder auf allen Geräten.

\begin{figure}[ht]
    \centfig{0.3}{facepalm.jpg}
    \caption{Facepalm \cite{fig:facepalm}}
\end{figure}

% By Alex E. Proimos - https://www.flickr.com/photos/proimos/4199675334/, CC BY 2.0, https://commons.wikimedia.org/w/index.php?curid=22535544

\subsection{Quellen und KI}

Das Erlernen der notwendigen Fertigkeiten und Wissen, um dieses Projekt zu erstellen, brauchte viel Zeit und einige Hilfsmittel. Neben allgemeinen Tutorials und Webanleitungen, waren auch generative KIs von grosser Hilfe beim Verwirklichen dieser Arbeit. Da Code sehr strukturiert ist und es offensichtlich viele Code Beispiele in den Trainingssets hat, war KI-generierter Code manchmal ein äusserst effizientes Hilfmittel.

Im Folgenden Abschnitt werden wir ein paar der wichtigsten Webseiten und Hilfsmittel, die wir verwendet haben vorstellen.

\subsubsection{Die Basics} 
Da unser Projekt als 'Full-Stack Webapplikation' bezeichnet werden kann, haben wir Tutorials gesucht, die React-Applicationen und zugehörige Server erklärt und demonstriert haben.

% 
\section{Methodik}
Unser Projekt folgte einem strukturierten Softwareentwicklungsprozess, der in mehrere Phasen unterteilt war. Wir begannen mit einer detaillierten Anforderungsanalyse und der Definition von Use Cases, um die Bedürfnisse der Endbenutzer zu verstehen.

\subsection{Backend-Entwicklung}
Die Implementierung von CRUD-Operationen und Authentifizierungssystemen stellte sicher, dass die Anwendung sicher und robust ist.

CRUD- und HTTP-Operationen:

CRUD steht für Create, Read, Update, Delete und bezieht sich auf die vier grundlegenden Operationen, die in vielen Anwendungen zur Interaktion mit Datenbanken oder Datenspeichern verwendet werden. Die Implementierung von CRUD-Operationen ermöglicht es Benutzern (oder Systemen), Daten zu erstellen, abzurufen (lesen), zu aktualisieren und zu löschen.

In einer Webanwendung werden CRUD-Operationen typischerweise über das Backend realisiert, wobei das Frontend (die Benutzeroberfläche) Anfragen an das Backend sendet, um Daten zu erstellen, abzurufen, zu aktualisieren oder zu löschen. Das Backend interagiert dann mit der Datenbank, um die angeforderten Aktionen durchzuführen und das Ergebnis (z.B. die abgerufenen Daten oder eine Bestätigung der Aktion) an das Frontend zurückzusenden, wo es dem Benutzer angezeigt wird.

Für das Backend, das oft eine API bereitstellt, können diese Operationen spezifischen Endpunkten entsprechen, z.B.:

•	GET /api/users	für das Abrufen von Benutzern (Read)

•	POST /api/user	für das Erstellen eines neuen Benutzers (Create)

•	PUT /api/user	für das Aktualisieren eines spezifischen Benutzers (Update)

•	POST /api/ex	für das Erstellen einer neuen Aufgabe (Create)

•	GET /api/ex	für das Abrufen von Aufgaben (Read)

•	PUT /api/ex/:id	für das Aktualisieren einer spezifischen Aufgabe (Update)

•	DELETE /api/ex/:id 	für das Löschen einer Aufgabe (Delete)

•	GET /api/ex/search/:serachText	für das Abrufen von Aufgaben nach Suchbegriffen (Read)

•	GET /api/ex/category/:category	für das Abrufen von Aufgaben nach Kategorien (Read)

•	GET /api/ex/:id	für das Abrufen einer spezifischen Aufgabe (Read)

•	POST /api/download/ 	für das Herunterladen der spezifischen Aufgabe im Warenkorb (Download)



\subsection{Authentifizierug und Autorisierung}
Argon2 nimmt als Eingabe das Passwort des Benutzers, einen Salz (eine zufällig generierte Zeichenfolge, die jedem Passwort-Hash hinzugefügt wird, um die Einzigartigkeit zu gewährleisten), und mehrere Parameter, die die Komplexität des Hashing-Vorgangs steuern (wie Speicherbedarf, Rechenzeit und Parallelität).Basierend auf diesen Eingaben führt Argon2 eine Reihe von komplexen, rechenintensiven Operationen durch. Diese Operationen sind so gestaltet, dass sie sowohl eine hohe Menge an CPU-Ressourcen als auch Speicher benötigen. Dies macht es sehr schwierig für Angreifer, Passwörter zu erraten oder Brute-Force-Angriffe durchzuführen, selbst wenn sie über leistungsfähige Hardware verfügen.

Das Ergebnis des Prozesses ist ein Passwort-Hash, der in der Datenbank gespeichert wird. Da der gleiche Prozess (mit dem gleichen Salz und denselben Parametern) immer denselben Hash erzeugt, kann das System das vom Benutzer bei der Anmeldung eingegebene Passwort überprüfen, ohne das eigentliche Passwort kennen oder speichern zu müssen. Sicherheit gegen Brute-Force-Angriffe: Die rechen- und speicherintensive Natur von Argon2 macht es teuer und zeitaufwändig, Hashes zu knacken. Dies schützt gegen Angriffe, die darauf abzielen, Passwörter durch Ausprobieren vieler möglicher Kombinationen (Brute-Force) zu erraten.

\subsection{Testing und Qualitätssicherung: }
Wir führten umfangreiche Tests durch, einschließlich Unit-Tests und Benutzerakzeptanztests, um sicherzustellen, dass die Anwendung den Erwartungen entspricht.


\subsection{Deployment und Inbetriebnahme: }
Nach Abschluss der Entwicklungsarbeiten wurde die Anwendung auf einer geeigneten Hosting-Plattform bereitgestellt und auf ihre Funktionalität in der Produktionsumgebung überprüft.


\section{Ergbenisse}
\subsection{Allgemeines}

Die entwickelte Webanwendung bietet Mathematiklehrern eine Plattform, auf der sie Aufgaben erstellen, durchsuchen, kommentieren und verwalten können. Zu den wichtigsten Funktionen gehören:

•	Aufgaben erstellen und verwalten: Lehrer können neue Aufgaben zur Datenbank hinzufügen und bestehende Aufgaben bearbeiten oder löschen.

•	Aufgabensuche: Die Suchfunktion ermöglicht es den Nutzern, spezifische Aufgaben nach Stichworten oder Kategorien zu finden.

•	Warenkorbsystem: Lehrer können ausgewählte Aufgaben in einem Warenkorb sammeln und daraus Aufgabenblätter erstellen.

Die Anwendung wurde erfolgreich implementiert und erfüllt die gestellten Anforderungen.





\newpage


\section{Fazit/Diskussion}


\newpage

\section{Zusammenfassung und Ausblick}
Als Abschlussarbeit war dieses Projekt ziemlich ambitioniert. Wir mussten uns dazu in zahlreiche Bereiche einlesen und uns neue Techniken aneignen. Ein grosser Knackpunkt waren die ganzen Schnittstellen zwischen den verschiedenen Modulen: Client, Server und Datenbank. Dort waren wir sehr froh, dass das Internet dazu viele Ressourcen liefert, die uns weiterhelfen konnten. 

Das Endprodukt erscheint uns als gelungenes Ergebnis für ein erstes Projekt in dieser Grösse. Natürlich haben wir in der Entwicklung und Umsetzung viele Fehler gemacht, die erfahrenen Webentwicklern nicht passiert wären. Es war aber auch die Absicht unsere Grenzen ein bisschen auszuloten, neue Erfahrungen zu machen und vor allem ein nützliches Tool am Schluss in der Hand zu haben. In dieser Hinsicht sind wir zufrieden.


\subsection{Ausblick}
In Zukunft könnten wir uns vorstellen noch weitere Funktionalitäten einzubauen. Insbesondere wäre eine hilfreiche Ergänzung, wenn Benutzer eine Kopie einer bestehenden Aufgabe erstellen könnten und eigene Änderungen machen könnten.

Andere mögliche Erweiterungen wären:
\begin{itemize}
    \item Einbauen einer Profilseite, wo Benutzerdaten eingesehen und geändert werden können. 
    \item Filter einbauen, der nur eigene Aufgaben anzeigt.
    \item Review Bereich, wo andere Nutzer Kommentare und Bewertungen zu Aufgaben machen können.
    \item Sortierung der Resultate nach Kriterien.
    \item Erstellen von 'Aufgabenlisten', die gespeichert werden können.
    \item Erweiterung auf andere Fachbereiche.
    \item Code-Struktur verbessern.
    \item \ldots
\end{itemize}


\subsection{Schlusswort}
Die Arbeit an diesem Projekt war eine wertvolle Erfahrung, die uns sowohl im Bereich der Softwareentwicklung als auch in der Projektorganisation weitergebracht hat. Die Erreichung unserer Ziele hat unsere Erwartungen erfüllt, und wir hoffen, dass unsere Anwendung einen praktischen Beitrag zur Verbesserung des Mathematikunterrichts leisten kann.

Wir möchten an dieser Stelle noch einmal allen danken, die uns in unserem Projekt unterstützt und geholfen haben und hoffen, dass dieses Tool auch in Zukunft eine Anwendung findet.


\begin{flushright}
    Basel, im Oktober 2024
    
    Oliver De Capitani und Patrick Weber
\end{flushright}

\newpage
   
  
% Hier beginnt der Anhang
\appendix
\section{Der Code}

Der vollständige Code ist unter \url{https://github.com/olidec/exercise-db-preact} abrufbar. 
\newpage





\section{Bilderverzeichnis}
\listoffigures

 

\newpage
\section{Quellen}

\renewcommand{\refname}{Quellenverzeichnis}

\begin{thebibliography}{11} 

\bibitem{fig:docker} De Capitani, Oliver {\em Graphische Übersicht über den Docker Container}, 2024, eigenes Bild

\bibitem{fig:workflow} De Capitani, Oliver {\em Ein Möglicher Workflow durch die Applikation}, 2024, eigenes Bild

\bibitem{fig:aufgabensuche} Weber, Patrick {\em Die Aufgabensuchkomponente}, 2024, eigenes Bild

\bibitem{fig:aufgabendetails} Weber, Patrick {\em Die Aufgabendetails}, 2024, eigenes Bild

\bibitem{fig:warenkorb} Weber, Patrick {\em Der Warenkorb}, 2024, eigenes Bild

\bibitem{fig:hinzufuegen} Weber, Patrick {\em Aufgaben Hinzufügen}, 2024, eigenes Bild

\bibitem{fig:editieren} Weber, Patrick {\em Aufgaben Editieren}, 2024, eigenes Bild

\bibitem{fig:all} Weber, Patrick {\em Komponenten mit Zugriff auf den SearchContext}, 2024, eigenes Bild

\bibitem{fig:all2} Weber, Patrick {\em Komponenten mit Zugriff auf den WarenkorbContext}, 2024, eigenes Bild

\bibitem{fig:trello} De Capitani, Oliver {\em Screenshot Trello}, 2024, eigenes Bild

\bibitem{fig:facepalm} Proimos, Alex E. {\em Caïn venant de tuer son frère Abel, by Henri Vidal in Tuileries Garden in Paris, France} 2012 (https://www.flickr.com/photos/proimos/4199675334/)



\end{thebibliography}








\end{document}