\section{Struktur der Webapplikation}

In der folgenden Graphik sehen wir schematisch die Struktur unserer Webapplikation: Das gesamte läuft in einem Docker Container. Ein Proxy (ein zwischengeschalteter Server) vermittelt zwischen dem User und den verschiedenen Bereichen des Docker Containers. Anfragen werden an den Client gesendet, der Client kommuniziert dann via Server mit der Datenbank.
\begin{figure}[ht]
    \centfig{0.6}{docker-container.png}
    \caption{Übersicht über den Docker Container \cite{fig:docker}}
\end{figure}

\subsection{Technische Details}

Der Docker Container läuft auf einem Linode Server mit der IP Adresse 172.104.225.119. Auf diesem Server ist eine Ubuntu Instanz geladen, die für die notwendige Infrastruktur sorgt. Damit man via regulärer URL auf den Server zugreifen kann, muss man in den Einstellungen einen DNS (Domain Name Server) Eintrag erstellen lassen. Damit lässt sich dann von einer Plaintext URL auf unsere Webapplikation zugreifen. 

Die Subdomains \verb|server.letstalkaboutx.ch| und \verb|prisma.letstalkaboutx.ch| haben wir auch registriert. Damit lassen sich die Anfragen an den Server (Login etc.) und direkter Zugriff auf die Datenbank realisieren.

Die Zertifikation läuft über selbstgezeichnete Zertifikate, die von \verb|certbot| automatisch erstellt werden. Somit kann immer via \verb|https| auf die Applikation zugegriffen werden.

Das Aufsetzen des externen servers wird in der Datei \verb|README.md| beschrieben.



\subsection{Die Applikation}

Wenn wir aus der oberen Graphik nur den inneren Teil betrachten, dann können wir ein bisschen näher auf die Funktionsweise unserer Webapplikation eingehen. Im unteren Bild sehen wir beispielhaft eine Anfrage vom Client an den Server, der dann wiederum bei der Datenbank anfragt.

\begin{figure}
    \centfig{0.6}{web-app-workflow.png}
    \caption{Ein Möglicher Workflow durch die Applikation \cite{fig:workflow}}
\end{figure}
Ein möglicher Ablauf wäre also:
\begin{enumerate}
    \item Client will die Aufgaben aus dem Warenkorb als \verb|.tex| Datei herunterladen und schickt dazu eine Liste von Aufgabennummern.
    \item Server verlangt von der Datenbank die Aufgaben, die der Client wünscht.
    \item Datenbank liefert die gewünschten Aufgaben an den Server.
    \item Server verarbeitet die Aufgaben zu einer \verb|.tex| Datei und liefert diese an den Client zurück.
\end{enumerate}

\subsection{Der Server}

Der Server erfüllt im Wesentlichen zwei Hauptaufgaben:
\begin{enumerate}
    \item Authentifizierung des Users: Testen, ob ein User die Berechtigung hat auf bestimmte Teile der Webseite zuzugreifen.
    \item Kommunikation mit der Datenbank: Anfragen vom User weiterleiten, die Antworten verarbeiten und in geeigneter Form an den User zurücksenden.
\end{enumerate}

Die einzelnen Routes werden durch den Express-Router verarbeitet. Somit können diese alle in separaten Dateien gespeichert werden und die Dateien werden übersichtlicher.

\subsubsection{Authentifizierung}
Die Authentifizierung läuft über das package \verb|passport.js|, das die notwendigen Tools für eine Registrierung und Authentifizierung liefert. Die Aufsetzung von \verb|passport.js| erfolgte in unserem Fall über eine sogenannte \verb|LocalStrategy| -- wir verwalten also unsere User selber. Es ist auch möglich bestehende Logins wie zum Beispiel Google oder Facebook in einer Seite auf ähnliche Weise einzubinden. 

Im Code unten ist ein Ausschnitt aus der Konfigurierung zu sehen, wo ein User verifiziert wird: Falls kein User mit dem eingegebenen \verb|username| existiert wird in der Callback-Funtion \verb|false| mitgegeben. Falls alles korrekt ist, kommt der User als Objekt in die Callback-Funktion. Diesen User kann man dann verwenden, um einen lokalen User zu setzen und ihm Berechtigungen zu geben.

Ebenfalls sehen wir, dass das Passwort nicht im Plaintext verwendet wird, sondern mit dem Package \verb|argon2| 'gehasht' wird. Die Verifizierung läuft dann über einen Vergleich von 'Hashes'. 

 \newpage

\begin{lstlisting}
const localStrategy = new LocalStrategy(
  { usernameField: "username", passwordField: "password" },
  verify
);
/**
 * Verify function for passport
 */
async function verify(username, password, callback) {
  try {
    const { user, success } = await getUser({ username: username });
    //check if user exists
    if (!success) {
      return callback(null, false);
    }

    //user exists check for valid pw
    if (await argon2.verify(user.password, password)) {
      return callback(null, user);
    }
    return callback(null, false);
  } catch (e) {
    return callback(e);
  }
}
\end{lstlisting}

Nachdem der User authentifiziert wurde, kann er dann auf die Funktionalitäten der Applikation zugreifen, wie z.B. Aufgaben suchen etc. Insbesondere kann ein User auch seine eigenen Aufgaben editieren oder löschen. Die \verb|UserId| wird von \verb|passport.js| in den lokalen Speicher geschrieben und wird gebraucht, um den Editierknopf anzuzeigen. 

\subsubsection{Kommunikation mit der Datenbank}

Der Server hat ebenfalls die Aufgabe mit der Datenbank zu kommunizieren. Hier lautet das Stichwort 'CRUD' -- 'Create, Read, Update, Delete'. Diese sind die Funktionalitäten, die der User in der Interaktion mit der Datenbank haben möchte:
\begin{enumerate}
    \item Aufgaben erstellen (Create)
    \item Aufgaben suchen resp. anschauen (Read)
    \item Eigene\footnote{Sowohl beim Editieren, wie auch beim Löschen von Aufagben wird die Identität des Users noch einmal auf dem Server verifiziert. Somit kann man nicht durch lokale Anpassung des Users Aufgaben in der Datenbank verändern.} Aufgaben editieren (Update)
    \item Eigene Aufgaben löschen (Delete)
\end{enumerate}

Um ein bisschen Struktur in diese Funktionalitäten zu bringen, sind diese Anfragen als separate Funktionen im Unterordner \verb|controllers| gespeichert. Dies erlaubt es Server Funktionen von Datenbank Funktionen zu trennen. Es handelt sich hier um Funktionen, mit unter anderem folgenden Aufgaben:
\begin{itemize}
    \item Kategorien aus der Datenbank lesen
    \item Aufgaben in die Datenbank schreiben
    \item Neue User erstellen
    \item etc.
\end{itemize}

Jede dieser Aufgaben entspricht dann einer Funktion, die an geeigneter Stelle aufgerufen werden kann\footnote{Somit bewegen wir uns in die Richtung der funktionellen Programmierung. Wir haben uns sicher nicht konsequent daran festgehalten, aber wo immer günstig, haben wir so gearbeitet. Dies gibt auch wieder eine grössere Übersicht und man kann einfacher Fehler finden.}. 

Da wir für die Datenbank Prisma verwenden, sind diese Anfragen wiederum als eingebaute Funktionen erhältlich und wir müssen einfach die korrekten Funktionen mit den gewünschten Eingaben aufrufen. 

Zum Beispiel liefert der untenstehende Controller die fünf neusten Aufgaben aus der Datenbank:

\newpage
\begin{lstlisting}
async function getRecentExercises() {
    try {
        const exercises = await prisma.exercise.findMany({
        take: 5,
        orderBy: {
            updatedAt: "desc",
        },
        });
        return exercises;
    } catch (error) {
        return { msg: "Error in DB request", err: error };
    }
}
\end{lstlisting}




\subsection{Client: Frontend-Architektur:} 

Im Frontend haben wir die verschiedenen Komponenten der Anwendung in eine Hierarchie gebracht, damit Änderungen an einer Komponente nicht ungewollt andere Teile der Applikation beeinflussen. 
Es handelt sich dabei um folgende Komponenten:

\begin{itemize}
  \item das Suchen der Aufgaben (AufgabenSuchen-Komponente)
  
  \item die detaillierten Ansicht einer Aufgabe (AufgabeDetails-Komponente)
\item das Anzeigen des Warenkorbsystems (Warenkorb-Komponente) 
  
  \item das Hinzufügen von Aufgaben (AufgabeHinzufügen-Komponente)
  
  \item das Editieren von Aufgaben (AufgabeEditieren-Komponente)
  

\end{itemize}

Diese Komponentenhierarchie führt zu einem besseren Überblick über die verschiedenen Funktionen und die Applikation wird in eine saubere und wartbare Codebasis gebracht.


\subsubsection{Die Komponentenhierarchie}

\texttt{3.4.1.1 Die AufgabenSuchen-Komponente:}
\begin{figure}[ht]
  \centfig{0.45}{Suchen_Komp.png}
  \caption{Die Aufgabensuchen-Komponente \cite{fig:aufgabensuche}}
\end{figure}


\begin{itemize}

\item Aufgabe (Card):
Grundbaustein einer Aufgabe, die den Inhalt, die Lösung, der Schwierigkeitsgrad, den Autor, die Kategorie und die Unterkategorie enthält.

\item Suchergebniss (SearchCard):
Komponente für die Anzeige eines Suchergebnisses, die eine einzelne Aufgabe (Card) darstellt, mit der Möglichkeit die Aufgabe zum Warenkorb hinzuzufügen bzw. aus dem Warenkorb zu löschen, die Details der Aufgabe (AufgabeDetails) anzeigen zu lassen und, sofern man Autor der Aufgabe ist, die Aufgabe zu editieren.

\item Auflistung aller Sucherergebnisse (SearchKorb):
Die einzelnen Aufgaben (SearchCard) werden in der Komponente SearchKorb zusammen angezeigt.
Diese Aufgaben kann man nach Sprache und Schwierigkeitsgrad filtern.

\item AufgabenSuchen (Seitenkomponente):
Diese Seitenkomponente umfasst zwei Komponenten (FindExSubCat und FindExBySearchText), die unterschiedliche Methoden zur Aufgabensuche anbieten. 

Mit FindExSubCat können Aufgaben nach Kategorien und Unterkategorien über eine API Anfrage \texttt{(askServer(`/api/ex?cat ={selectedCategory}\&subcat=\\{subcategoryName}`,"GET"))} gesucht werden (siehe Kapitel 3.4.5 Zustandsmanagement für die Komponente FindExSubCat).

Mit FindExBySearchText kann eine Aufgabe nach einer textbasierten Suchanfrage (inputValue) gesucht werde, mit der API Anfrage \texttt{askServer(`/api/ex?search=\\\${inputValue}`,"GET")}.

Beides Mal werden die anzuzeigenden Aufgaben in der Komponente SearchKorb angezeigt.


\end{itemize}

\texttt{3.4.1.2 Die AufgabeDetails-Komponente:}

\begin{figure}[H]
  \centfig{0.35}{AufgDetails.png}
  \caption{Die Aufgabendetails \cite{fig:aufgabendetails}}
\end{figure}

\begin{itemize}
\item Details einer Aufgabe (AufgabeDetails):
Eine Detailansicht für jede Aufgabe erfolgt in einer Modal Ansicht (Fenster innerhalb eines Browser Fensters erscheint), wobei neben dem Aufgabentext auch die Lösung, die Kategoriezugehörigkeit, der Schwierigkeitsgrad, die Sprache und der Autor der Aufgabe zu sehen ist. 
Die API Anfrage \texttt{askServer("/api/ex?id=\${id}", "GET")} ruft von der Datenbank die Aufgabe mit der entsprechenden ID ab.
Die Aufgabe kann auch in dieser detaillierten Ansicht zum Warenkorb hinzugefügt werden oder aus dem Warenkorb gelöscht werden.



\end{itemize}

\texttt{3.4.1.3 Die Warenkorb-Komponente:}


\begin{figure}[H]
  \centfig{0.35}{Warenkorb.png}
  \caption{Der Warenkorb \cite{fig:warenkorb}}
\end{figure}

\begin{itemize}

\item Aufgabe im Warenkorb (WarenCard):
Eine Aufgabe (Card) im Warenkorb kann gelöschte werden und ihre Position mit der vorhergehenden oder nachfolgenden Aufgabe vertauscht werden, um die richtige Reihenfolge für das herunterladen der Aufgabe zu erhalten. 


\item Warenkorb (Seitenkomponente):
Die einzelnen Aufgaben (WarenCard) werden untereinander im Warenkorb angezeigt und über einen Button können alle Aufgabe im Warenkorb heruntergeladen werden.


\end{itemize}

\texttt{3.4.1.4 Die AufgabeHinzufügen-Komponente:}
\begin{figure}[H]
  \centfig{0.2}{ExForm.png}
  \caption{Aufgaben Hinzufügen \cite{fig:hinzufuegen}}
\end{figure}

\begin{itemize}

\item{Formularfelder einer Aufgabe (ExerciseForm)}:
Diese Komponente umfasst Formularfelder für die Eingabe von Sprache, Schwierigkeitsgrad, Kategorie, Unterkategorie, Aufgabentext und Lösung.

\item ExForm (Seitenkomponente): :
Diese Komponente ermöglicht das Erstellen neuer Aufgaben. Über eine API-Anfrage \texttt{(askServer(/api/ex, "POST", ex))} werden die Daten an das Backend gesendet und die Aufgabe wird, sofern alle Formularfelder richtig ausgefüllt sind, in die Datenbank hinzugefügt.


\end{itemize}

\texttt{3.4.1.5 Die AufgabeEditieren-Komponente:}
\begin{figure}[H]
  \centfig{0.2}{EditForm.png}
  \caption{Aufgaben Editieren \cite{fig:editieren}}
\end{figure}

\begin{itemize}
\item{EditForm} (Seitenkomponente):
Die Sprache, der Schwierigkeitsgrad, die Kategorie, die Unterkategorie, der Aufgabentext und die Lösungen können über Formularfelder aktualiert werden, sofern man Autor der Aufgabe ist.
Die API Anfrage \texttt{askServer("/api/ex", "PUT", exWithCategory)} aktualisiert die Aufgabe in der Datenbank.


\end{itemize}
\subsubsection{Weiteren Seitenkomponenten}

\begin{itemize}
\item Startseite:
 Die Startseite (Landingpage) zeigt eine zusammenfassende und einladende Beschreibung der Webseite an. Über einen Registeren/Login-Button gelangt man zur Loginseite. Hat man noch keine Login, kann man über einen Link zur Registrierungsseite wechseln.

\item Login-Komponente:
Diese Login-Komponente ermöglicht es dem Benutzer, seinen Benutzernamen und sein Passwort in ein Formular einzugeben. Die API Anfrage \texttt{(askServer("/login", "POST", {username,password}))} sendet die Daten an das Backend.
Beim Absenden des Formulars wird ein Cookie mit den Authentifizierungsdaten gesendet. Dieses Cookie wird in der Anwendung gespeichert und geprüft, um den Zugang zu geschützten Bereichen der Anwendung zu gewähren.


\item Registrieren-Komponente:
Die Registrieren-Komponente ermöglicht es dem Benutzer über die Email Adresse, den Benutzernamen und sein Passwort, einen neuen Benutzer anzulegen. Die API Anfrage \texttt{(askServer("/register", "POST", user))} sendet die Daten an das Backend.

\item Navigations-Komponente:
Die Menu-Komponente stellt die Hauptnavigation der Anwendung bereit und ermöglicht den Zugriff auf die verschiedenen Seiten und Funktionen der App, darunter die Startseite (/), das Hinzufügen von Aufgaben (/add), das Finden von Aufgaben (/find), den Warenkorb (/warenkorb), Infos zur Webseite (/about), die Registrierung(/register) und der Login (/login).


\end{itemize}



\subsubsection{Zustandsmanagement mit useContext } 

Ein effektives Zustandsmanagement ist entscheidend für die Verwaltung der Anwendungsdaten sowie der verschiedenen UI-Zustände, wie Benutzereingaben oder geladene Daten (gesuchte Aufgaben). Preact bietet uns hierfür mehrere Optionen: Für lokale Zustände innerhalb einzelner Komponenten nutzen wir die useState und useEffect Hooks sowie Signals (Kapitel 3.4.4), die eine einfache und reaktive Verwaltung dieser Zustände ermöglichen. 

Für komplexe Szenarien, in denen ein globaler Zustand erforderlich ist – also Daten, die von mehreren Komponenten und deren Unterkomponenten (Children) gleichzeitig genutzt werden – greifen wir auf die useContext API zurück. Mit \texttt{useContext} können wir eine zentrale Datenquelle erstellen, die es verschiedenen Komponenten ermöglicht, auf dieselben Daten oder Funktionen zuzugreifen und diese zu aktualisieren, was die Anwendung sowohl leistungsfähig als auch flexibel macht.

Insgesammt wurde zwei zentrale Datenquellen (Kontext-Provider), SearchContext und WarenkorbContext, erstellt. 

\texttt{3.4.3.1 Der SearchContext}

Der SearchContext stellt sicher, dass wichtige Zustandsinformationen wie Suchergebnisse, ausgewählte Kategorien und Benachrichtigungen zentral gespeichert und zwischen verschiedenen Komponenten geteilt werden können. 


\begin{lstlisting}[language=Python]
const { showNotification, cartSearch, setCartSearch, searchText, categor, deleteCard } = useContext(SearchContext);
 \end{lstlisting}  
 
Hier werden die Variablen und Funktion (\texttt{showNotification, setCartSearch, cartSearch, searchText, categor und deleteCard}) aus dem SearchContext abgerufen, die als Signals (Kapitel 3.4.4) implementiert sind und im localStorage des Browsers gespeichert werden, da diese Daten zwischen verschiedenen Komponenten geteilt werden müssen.

\begin{itemize}
  \item \texttt{showNotification} speichert und verwaltet Benachrichtigungen oder Fehlermeldungen, die beim Bearbeiten, Eingeben oder Suchen von Aufgaben auftreten und zeigt sie in roter oder grüner Farbe für eine kurze Zeit an.
  \item \texttt{cartSearch} speichert die Suchergebnisse in einem Array und verwaltet diese.
    \item \texttt{setCartSearch} speichert die Suchergebnisse in der Variablen cartSearch, die von der Benutzerabfrage zurückgegeben werden.
    \item \texttt{searchText} enthält den eingegebenen Suchtext nach dem der Benutzer die Aufgaben sucht und zeigt ihn oberhalb der Suchergebnisse an.
    \item \texttt{categor} speichert die vom Benutzer angeklickte Kategorie und Subkategorie und zeigt sie oberhalb der Suchergebnisse an.
    \item \texttt{deleteCard} entfernt eine Aufgabe aus der Datenbank, sofern man Autor der Aufgabe ist.
  
\end{itemize}


Folgende Grafik zeigt, an welche Komponenten (und immer auch alle untergeordneten Komponenten (Children)) der SearchContext die verschiedene Variablen und Funktionen weitergibt:
\begin{figure}[H]
\centfig{0.3}{all.png}
\caption{Komponenten mit Zugriff auf den SearchContext \cite{fig:all}}
\end{figure}


\texttt{3.4.3.2 Der WarenkorbContext}

Der WarenkorbContext stellt sicher, dass Zustandinsformationen zwischen dem Warenkorb und dem Anzeigen der gesuchten Aufgaben zentral gespeichert und zwischen Komponenten geteilt werden können.

\begin{lstlisting}[language=Python]
const { cartItems, addToKorb, handleDelete, getIndex, getCartCount } = useContext(WarenkorbContext);
 \end{lstlisting} 

 \begin{itemize}

  \item \texttt{addToKorb} speichert die Aufgabe in der Warenkorb-Liste.
  \item \texttt{handleDelete} entfernt eine Aufgabe aus der Warenkorb-Liste.
  \item  \texttt{cartItems} speichert die ausgewählten Aufgaben für dem Warenkorb in der Warenkorb-Liste und verwaltet diese.
  \item \texttt{getIndex} gibt die Position einer Aufgabe in der Warenkorb-Liste an. (Ist der Index -1, ist die Aufgabe nicht im Warenkorb und der Button addToKorb wird angezeigt.)
  \item \texttt{getCartCount} gibt die Anzahl der gespeicherten Aufgaben in der Warenkorb-Liste an.
 \end{itemize}

 Folgende Grafik zeigt, an welche Komponenten (und immer auch alle untergeordneten Komponenten (Children)) der WarenkorbContext die verschiedene Variablen und Funktionen weitergibt:
 \begin{figure}[ht]
 \centfig{0.3}{all2.png}
 \caption{Komponenten mit Zugriff auf den WarenkorbContext \cite{fig:all2}}
 \end{figure}


 \subsubsection{Zustandsmanagement mit useState, useEffect und Signals}

 Der useState Hook ermöglicht es, Zustände innerhalb einer Funktionskomponente zu verwalten. Wenn man useState aufruft, erhält man ein Array mit zwei Werten: den aktuellen Zustand und eine Funktion, um diesen Zustand zu aktualisieren.
 
 \begin{lstlisting}
 const [state, setState] = useState(initialValue);
 // state: Die aktuelle Zustandsvariable, den man verwenden kann.
 // setState: Eine Funktion, mit der man den Zustand der Variable aktualisieren kann.
 //initialValue: Der Anfangswert des Zustands.
 
 \end{lstlisting}
 
 
 Der useEffect Hook wird verwendet, um Seiteneffekte (dependencys) in Funktionskomponenten zu behandeln. Ein Seiteneffekt ist z.B. das Abrufen von Daten von einem Server oder das Ändern des DOM. Sofern es eine Änderung der Variablen (dependencys) in diesem Array gibt, wird die useEffect()-Funktion erneut ausgeführt.
 
 \begin{lstlisting}
 useEffect(() => {
  // Effekt-Funktion, z.B. Daten laden oder DOM aktualisieren
   return () => {
     // Cleanup-Funktion, z.B. Event-Listener entfernen
   };
 }, [dependency1, dependency2]);
 
 \end{lstlisting}
   
 
 
Zusätzlich bietet Preact die Möglichkeit, Signals zu verwenden, die eine noch feinere Kontrolle über den reaktiven Datenfluss bieten.
Signals sind reaktive Zustandscontainer, die es ermöglichen, Zustandsänderungen automatisch zu verfolgen und so Zustände innerhalb der Komponente und zwischen den Komponenten zu aktualisieren. Komponenten, die ein Signal lesen, werden automatisch neu gerendert, wenn sich das Signal ändert.
 

\subsubsection{Code-Beispiel: Zustandsmanagement für die Komponente FindExSubCat } 
Das folgende Beispiel beziehen sich auf die Komponente \texttt{FindExSubCat} um Aufgaben nach Kategorien und Subkategorien durch clicken auf dieselben zu suchen.

Das Signal \texttt{cat} wird verwendet um alle Kategorien vom Server zu laden.

\begin{lstlisting}
const cat = signal([]);
const loadCat = async () => {
  const res = await askServer("/api/cat/", "GET");
  cat.value = res.response;
};

\end{lstlisting}

Nun wird die useEffect()-Funktion verwendet damit beim laden der Seite die loadCat()-Funktion die Kategorien vom Server lädt und sie in das Array categories schreibt \\\texttt{(setCategories(cat.value))}. Da sich diese Variable beim laden der Seite verändern könnte, wird sie als useState Variable definiert. 
Diese useEffect()-Funktion wird nur beim laden der Seite ausgeführt, da die Liste der Abhängigkeiten (dependencys, leeres Array auf der letzten Zeile) leer bleibt. 

\begin{lstlisting}
import { cat, loadCat } from "../signals/categories.js";
const [categories, setCategories] = useState([]);
useEffect(() => {
    const fetchCategories = async () => {
    await loadCat();
    setCategories(cat.value);
    };
    fetchCategories();
  }, []);

\end{lstlisting}


Die Funktion \texttt{onCategoryClick} wird aufgerufen, wenn eine Kategorie angeklickt wird. Sie setzt die ausgewählte Kategorie in die Variable \texttt{selectedCategory} und setzt das Array der Unterkategorie auf einen leeren String zurück. Dann sendet die Funktion eine Anfrage an den Server, um Übungen basierend auf der Kategorie zu laden und in der Variablen \texttt{excat} zu speichern.

\begin{lstlisting}[language=Python]
const [selectedCategory, setSelectedCategory] = useState("");
const [selectedSubcategory, setSelectedSubcategory] = useState("");

const onCategoryClick = async (categoryName) => {
setSelectedCategory(categoryName);
setSelectedSubcategory(""); # Reset subcategory when selecting a new category
const route = `/api/ex?cat=${categoryName}`;
const res = await askServer(route, "GET");
const excat = res.response;
    
\end{lstlisting}

 Wenn der Server keine Antwort auf die Anfrage gibt (Status ungleich 200) oder keine Übungen zurückgibt, wird eine Benachrichtigung (showNotification) angezeigt und das Array der gesuchten Aufgaben sowie die Kategorieauswahl zurückgesetzt. Andernfalls wird die Liste der gefundenen Übungen mit der Funktion \texttt{setCartSearch} in die Variablen \texttt{cartSearch} gespeichert. (siehe Kapitel 3.4.3.1 SearchContext)


\begin{lstlisting}[language=Python]
    if (res.status != 200 || excat.length === 0) {
      showNotification("No exercise matches the search term.", "red");
      setCartSearch([]);
      setSelectedCategory("");
      searchText.value = "";
      categor.value[0] = categoryName;
      categor.value[1] = "";
    } else {
      setCartSearch(excat);
      searchText.value = "";
      categor.value[0] = categoryName;
      categor.value[1] = "";
    }
  };


\end{lstlisting}



Die useEffect() Funktion für Unterkategorien sucht die Unterkategorien der ausgewählten Kategorie, die in der Variablen \texttt{selectedCategory} gespeichert wurde und speichert diese im Array \texttt{subcategories} und zwar jedes Mal, wenn sich \texttt{selectedCategory} oder \texttt{categories} ändert (Seiteneffekt). 


\begin{lstlisting}[language=Python]
const [subcategories, setSubcategories] = useState([]);
useEffect(() => {
    if (selectedCategory) {
      const category = categories.find((c) => c.name === selectedCategory);
      setSubcategories(category ? category.subcategories : []);
    } else {
      setSubcategories([]);
    }
  }, [selectedCategory, categories]);
      
\end{lstlisting}



Die Funktion \texttt{onSubcategoryClick} wird aufgerufen, wenn eine Unterkategorie ausgewählt wird. Sie sendet eine Anfrage an den Server, um Übungen basierend auf der ausgewählten Unterkategorie zu laden und und in der Variablen \texttt{exsubcat} zu speichern.

\begin{lstlisting}[language=Python]
const onSubcategoryClick = async (subcategoryName) => {
    setSelectedSubcategory(subcategoryName);

    const route = `/api/ex?cat=${selectedCategory}&subcat=${subcategoryName}`;
    const res = await askServer(route, "GET");
    const exsubcat = res.response;
    
    
\end{lstlisting}

 Ähnlich wie bei der Kategorieauswahl wird eine Benachrichtigung angezeigt und die Unterkategorieauswahl zurückgesetzt, wenn keine passenden Übungen gefunden werden. Ansonsten wird die Liste der Übungen mit der Funktion \texttt{setCartSearch} gespeichert. (siehe Kapitel 3.4.3.1 SearchContext).


\begin{lstlisting}[language=Python]
if (res.status != 200 || exsubcat.length === 0) {
      showNotification("No exercise matches the search term.", "red");
      setCartSearch([]);
      searchText.value = "";
      categor.value[1] = subcategoryName;
    }
else {
      setCartSearch(exsubcat);
      searchText.value = "";
      categor.value[1] = subcategoryName;
    }
  };
\end{lstlisting}

\subsection{Datenbank}

\subsubsection{Datenbankstruktur}

In diesem Abschnitt wird das Datenbankmodell beschrieben, welches für die Speicherung der Benutzerinformationen, Aufgaben und deren Kategorisierung verwendet wird. Die Implementierung erfolgt mit Prisma ORM und PostgreSQL als Datenbank. Im Folgenden wird die Struktur der relevanten Tabellen dokumentiert.

Das Datenbankmodell umfasst vier Haupttabellen: \texttt{User}, \texttt{Exercise}, \texttt{Category} und \texttt{Subcategory}. Diese Tabellen sind miteinander verknüpft, sodass Benutzer Aufgaben erstellen und diese in Kategorien und Unterkategorien organisieren können.

\newpage
\begin{lstlisting}[language=Python]
  generator client {
  provider        = "prisma-client-js"
  previewFeatures = ["fullTextIndex", "fullTextSearch"]
}

datasource db {
  provider = "postgresql"
  url      = env("DATABASE_URL")
}

model User {
  id        Int        @id @default(autoincrement())
  email     String     @unique
  username  String     @unique 
  password  String?
  role      String     @default("USER")
  exercises Exercise[]
  retry     Int        @default(0)
  retryExp  DateTime?  
}

model Exercise {
  id            Int         @id @default(autoincrement())
  createdAt     DateTime    @default(now())
  updatedAt     DateTime    @updatedAt
  summary       String? // Optional, kann null sein
  content       String
  solution      String
  language      String      @default("Deutsch")
  difficulty    Int         @default(1)
  authorId      Int         @default(1)
  author        User        @relation(fields: [authorId], references: [id])
  categoryId    Int         @default(1)
  categories    Category    @relation(fields: [categoryId], references: [id])
  subcategoryId Int @default(1)
  subcategories Subcategory @relation(fields: [subcategoryId], references: [id])
}

model Category {
  id            Int           @id @default(autoincrement())
  name          String
  subcategories Subcategory[]
  exercises     Exercise[]
}

model Subcategory {
  id         Int        @id @default(autoincrement())
  name       String
  categoryId Int
  category   Category   @relation(fields: [categoryId], references: [id], onDelete: Cascade)
  exercises  Exercise[]
}

\end{lstlisting}




\subsubsection{Tabelle \texttt{User}}

Die Tabelle \texttt{User} speichert die Informationen der Benutzer. Jeder Benutzer hat eine eindeutige ID sowie eine E-Mail-Adresse und einen Benutzernamen. Optional kann ein Passwort gespeichert werden. Benutzer können eine Rolle besitzen (standardmäßig \texttt{USER}), die ihre Berechtigungen festlegt. Zusätzlich verwaltet die Tabelle die Aufgaben, die ein Benutzer erstellt hat.

\begin{itemize}
  \item \texttt{id}: Primärschlüssel, automatisch inkrementiert.
  \item \texttt{email}: Eindeutige E-Mail-Adresse des Benutzers.
  \item \texttt{username}: Eindeutiger Benutzername.
  \item \texttt{password}: Optionales Passwortfeld, kann null sein.
  \item \texttt{role}: Rolle des Benutzers (standardmäßig \texttt{USER}).
  \item \texttt{exercises}: Beziehung zur Tabelle \texttt{Exercise}. Verknüpfung zu den von diesem Benutzer erstellten Aufgaben, dies ist eine \texttt{1:n}-Beziehung da der Benutzer mehrer Aufgaben erstellen kann
  \item \texttt{retry}: Anzahl der fehlgeschlagenen Anmeldeversuche.
  \item \texttt{retryExp}: Zeitpunkt, ab dem der Benutzer nach mehreren Fehlversuchen erneut versuchen kann, sich anzumelden.
\end{itemize}

\subsubsection{Tabelle \texttt{Exercise}}

Die Tabelle \texttt{Exercise} speichert die einzelnen Mathematikaufgaben, die von Benutzern erstellt werden. Jede Aufgabe enthält Informationen wie den Inhalt, die Lösung, die Sprache und den Schwierigkeitsgrad. Jede Aufgabe ist einem Benutzer (dem Autor), einer Kategorie und einer Unterkategorie zugeordnet.

\begin{itemize}
  \item \texttt{id}: Primärschlüssel, automatisch inkrementiert.
  \item \texttt{createdAt}: Erstellungsdatum der Aufgabe, standardmäßig die aktuelle Zeit.
  \item \texttt{updatedAt}: Automatisch aktualisiertes Datum, wenn die Aufgabe geändert wird.
  \item \texttt{summary}: Eine optionale Zusammenfassung der Aufgabe.
  \item \texttt{content}: Der eigentliche Inhalt der Aufgabe.
  \item \texttt{solution}: Die Lösung der Aufgabe.
  \item \texttt{language}: Sprache der Aufgabe (standardmäßig \texttt{Deutsch}).
  \item \texttt{difficulty}: Schwierigkeitsgrad der Aufgabe (standardmäßig 1).
  \item \texttt{authorId}: Fremdschlüssel, der den Benutzer referenziert, der die Aufgabe erstellt hat.
   \item \texttt{author}: Beziehung zur Tabelle \texttt{User}, der die Aufgabe erstellt hat. Diese Beziehung ist eine \texttt{1:n}-Beziehung, da ein Benutzer mehrere Aufgaben erstellen kann.
  \item \texttt{categoryId}: Fremdschlüssel, der die Kategorie referenziert, zu der die Aufgabe gehört.
  
\item \texttt{categories}: Beziehung zur Tabelle \texttt{Category}. Diese Beziehung ist eine \texttt{n:1}-Beziehung, da jede Aufgabe genau einer Kategorie zugeordnet ist, aber eine Kategorie mehrere Aufgaben enthalten kann.
  \item \texttt{subcategoryId}: Fremdschlüssel, der die Unterkategorie referenziert, zu der die Aufgabe gehört.
    \item \texttt{subcategories}: Beziehung zur Tabelle \texttt{Subcategory}. Diese Beziehung ist ebenfalls eine \texttt{n:1}-Beziehung, da jede Aufgabe genau einer Unterkategorie zugeordnet ist, aber eine Unterkategorie mehrere Aufgaben umfassen kann.
\end{itemize}

\subsubsection{Tabelle \texttt{Category}}

Die Tabelle \texttt{Category} enthält Informationen über die Hauptkategorien, denen Aufgaben zugeordnet werden können. Eine Kategorie kann mehrere Unterkategorien und Aufgaben beinhalten.

\begin{itemize}
  \item \texttt{id}: Primärschlüssel, automatisch inkrementiert.
  \item \texttt{name}: Name der Kategorie.
   \item \texttt{subcategories}: Beziehung zur Tabelle \texttt{Subcategory}. Diese Beziehung ist eine \texttt{1:n}-Beziehung, da eine Kategorie mehrere Unterkategorien enthalten kann, aber jede Unterkategorie nur zu einer Kategorie gehört.
  \item \texttt{exercises}: Beziehung zur Tabelle \texttt{Exercise}. Diese Beziehung ist ebenfalls eine \texttt{1:n}-Beziehung, da eine Kategorie mehrere Aufgaben enthalten kann, aber jede Aufgabe nur zu einer Kategorie gehört.
\end{itemize}

\subsubsection{Tabelle \texttt{Subcategory}}

Die Tabelle \texttt{Subcategory} speichert Informationen über die Unterkategorien, die einer übergeordneten Kategorie zugeordnet sind. Jede Unterkategorie kann mehrere Aufgaben enthalten.

\begin{itemize}
  \item \texttt{id}: Primärschlüssel, automatisch inkrementiert.
  \item \texttt{name}: Name der Unterkategorie.
  \item \texttt{categoryId}: Fremdschlüssel, der die übergeordnete Kategorie referenziert.
 \item \texttt{category}: Beziehung zur Tabelle \texttt{Category}. Diese Beziehung ist eine \texttt{n:1}-Beziehung, da jede Unterkategorie zu genau einer Kategorie gehört, aber eine Kategorie mehrere Unterkategorien enthalten kann.Beziehung zur übergeordneten Kategorie, bei deren Löschung die Unterkategorie ebenfalls gelöscht wird (Cascade-Löschung).
 
   \item \texttt{exercises}: Beziehung zur Tabelle \texttt{Exercise}. Diese Beziehung ist eine \texttt{1:n}-Beziehung, da eine Unterkategorie mehrere Aufgaben enthalten kann, aber jede Aufgabe nur zu einer Unterkategorie gehört.
\end{itemize}

\subsubsection{Zusammenfassung}

Das beschriebene Datenbankmodell ermöglicht eine klare und logische Strukturierung von Benutzern, Aufgaben, Kategorien und Unterkategorien. Mithilfe von \texttt{Prisma} ORM und \texttt{PostgreSQL} als Datenbank wird sichergestellt, dass die Anwendung skalierbar und effizient auf große Datenmengen zugreifen kann. Jede Tabelle ist durch eindeutige Beziehungen verknüpft, was eine einfache Verwaltung der Daten und eine effiziente Abfrage ermöglicht.


Die Datenbankstruktur unterstützt umfassende Interaktionen zwischen Benutzern, ihren Aufgaben sowie die Kategorisierung von Aufgaben.

