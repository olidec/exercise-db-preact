\subsection{Koordination}

In einer ersten Phase haben wir Trello benutzt, um einen Überblick über die notwendigen Komponenten und Teilschritte zu bekommen. Trello ist ein visuelles Tool, um den Arbeitsprozess eines Projekts zu planen und zu koordinieren. Gerade zu Beginn sah man oft vor lauter Bäumen den Wald nicht -- da war ein solches Tool wirklich hilfreich. Das untenstehende Bild zeigt exemplarisch einen Zustand, wie das Tool eingesetzt wurde. Aktive Tasks werden von links nach rechts verschoben, sobald sie bearbeitet oder erledigt werden.
\begin{figure}[ht]
    \centfig{0.6}{trello.png}
    \caption{Screenshot Trello \cite{fig:trello}}
\end{figure}

Mit der Zeit bekamen wir eine bessere Übersicht über die Struktur und notwendigen Komponenten der Webapplikation und wir sahen uns nicht mehr so gezwungen, das Tool zu verwenden. Als eine Zweiergruppe, brauchte es auch immer nur einen einzigen Anruf oder Kurznachricht, um das ganze Team auf denselben Wissensstand zu bringen.