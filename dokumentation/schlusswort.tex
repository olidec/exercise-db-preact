\section{Zusammenfassung und Ausblick}
Als Abschlussarbeit war dieses Projekt ziemlich ambitioniert. Wir mussten uns dazu in ganz viele Bereiche einlesen und uns neue Techniken aneignen. Ein grosser Knackpunkt waren die ganzen Schnittstellen zwischend den verschiedenen Modulen: Client, Server und Datenbank. Dort waren wir sehr froh, dass das Internet dazu viele Ressourcen liefert, die uns weiterhelfen konnten. 

Das Endprodukt erscheint uns als gelungenes Ergebnis für ein erstes Projekt in dieser Grösse. Natürlich haben wir in der Entwicklung und Umsetzung viele Fehler gemacht, die erfahrenen Webentwicklern nicht passiert wären. Es war aber auch die Absicht unsere Grenzen ein bisschen auszuloten, neue Erfahrungen zu machen und vor allem ein nützliches Tool am Schluss in der Hand zu haben. In dieser Hinsicht sind wir zufrieden.


\subsection{Ausblick}
In Zukunft könnten wir uns vorstellen noch weitere Funktionalitäten einzubauen. Insbesondere wäre eine hilfreiche Ergänzung, wenn Benutzer eine Kopie einer bestehenden Aufgabe erstellen könnten und eigene Änderungen machen könnten.

Andere mögliche Erweiterungen wären:
\begin{itemize}
    \item Einbauen einer Profilseite, wo Benutzerdaten eingesehen und geändert werden können. 
    \item Filter einbauen, der nur eigene Aufgaben anzeigt.
    \item Review Bereich, wo andere Nutzen Kommentare und Bewertungen zu Aufgaben machen können.
    \item Sortierung der Resultate nach Kriterien.
    \item Erstellen von 'Aufgabenlisten', die gespeichert werden können.
    \item Erweitern auf andere Fachbereiche.
    \item Code Struktur verbessern.
    \item \ldots
\end{itemize}


\subsection{Schlusswort}
Die Arbeit an diesem Projekt war eine wertvolle Erfahrung, die uns sowohl im Bereich der Softwareentwicklung als auch in der Projektorganisation weitergebracht hat. Die Erreichung unserer Ziele hat unsere Erwartungen erfüllt, und wir hoffen, dass unsere Anwendung einen praktischen Beitrag zur Verbesserung des Mathematikunterrichts leisten kann.

Wir möchten an dieser Stelle noch einmal allen danken, die uns in unserem Projekt unterstützt und geholfen haben und hoffen, dass dieses Tool auch in Zukunft eine Anwendung findet.


\begin{flushright}
    Basel, im Oktober 2024
    
    Oliver De Capitani und Patrick Weber
\end{flushright}