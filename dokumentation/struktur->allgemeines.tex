\subsection{Allgemeines}
Die Digitalisierung hat das Bildungswesen grundlegend transformiert, insbesondere durch den Einsatz von E-Learning-Plattformen und digitalen Hilfsmitteln. In der Fachliteratur finden sich zahlreiche Ansätze zur Entwicklung von Webanwendungen, die den Unterricht effizienter gestalten sollen. Um dieses Ziel zu erreichen, haben wir uns für den Einsatz moderner und bewährter Web-Technologien entschieden.

Preact wurde als unsere bevorzugte Frontend-Bibliothek ausgewählt, da es eine leichtgewichtige und performante Alternative zu React darstellt. Preact ermöglicht uns die Entwicklung interaktiver Benutzeroberflächen, die eine reibungslose und schnelle Benutzererfahrung gewährleisten. Die Bibliothek bietet die Möglichkeit, wiederverwendbare Komponenten zu erstellen, was die Entwicklung und Wartung der Benutzeroberfläche erheblich vereinfacht.

Für das Backend haben wir Express gewählt, ein minimalistisch und flexibel gestaltetes Framework für Node.js. Express bietet eine robuste Basis für die Entwicklung von serverseitiger Logik und API-Routen. Es ermöglicht uns, HTTP-Anfragen effizient zu verarbeiten, Daten sicher zu speichern und verschiedene Authentifizierungsmechanismen zu implementieren. Dank der modularen Struktur von Express konnten wir eine skalierbare Backend-Architektur entwickeln, die sowohl leistungsfähig als auch leicht zu erweitern ist.

Die Datenbankinteraktion wird durch Prisma gesteuert, ein modernes ORM (Object-Relational Mapping) Tool, das eine einfache und typsichere Datenbankanbindung ermöglicht. Prisma erlaubt es uns, die Datenbankabfragen direkt in TypeScript zu schreiben, was die Konsistenz und Sicherheit des Codes erhöht. Es unterstützt Migrationen, wodurch das Datenbankschema während der Entwicklung problemlos angepasst werden kann, und ermöglicht eine effiziente Verwaltung der Datenmodelle.

Unsere Hypothese lautet, dass durch die gezielte Auswahl und Kombination dieser Technologien – Preact, Express und Prisma – eine leistungsstarke, skalierbare, benutzerfreundliche und effiziente Webanwendung entwickelt werden kann, die den Anforderungen von Mathematiklehrern gerecht wird.